
\chapter{Méthodes de travail}

Dans le cadre de notre projet SPID de l’UV 3.4, nous avons mis en place des méthodes de travail pour ne pas être dépassé par la charge des actions à mener et palier ainsi aux éventuels obstacles.  
L’organisation au sein d’une équipe, pour la réalisation d’un projet en commun, est primordiale pour éviter les échecs.

Nous avons utilisé les premières séances pour apprendre à connaitre chaque membre de l’équipe : connaitre ses méthodes de travail, son rythme, sa tendance à prendre des initiatives ou à suivre des idées, son expérience de leader dans son cursus scolaire, ses disponibilités dans la semaine pour d’éventuelles séances supplémentaires de projet. 
Etant une équipe constituée de 6 membres, nous avons constitué deux sous équipes. Nous sommes tous parties sur la mise en place des besoins, fonctions et exigences pendant les premières semaine du projet. 

Un premier groupe est partie sur l'état de l'art des systèmes de détection des pupilles et le traitement. La seconde équipe se charge de la recherche, l'actualisation et le raffinement  des exigences. 

Les tâches sont réparties au sein de chaque groupe après un point de début de séance fixant les objectifs journaliers. 

\chapter{Outils pour les échanges}

Malgré un travail en groupe, chacun fait ses propres recherches avec à la clef un compte rendu de son activité. Pour échanger nos avancées et travaux sur le projet, nous utilisations Git Hub. Après les séances "atelier" de formation à Git Hub, le groupe utilise au mieux se moyen d'échange pour suivre le projet ainsi que les avancées de chaque membres grâce aux graphes de réseaux pour les dépôts ou modifications. 

En dehors du projet, pour les questions de logistiques (horaires de travail, lieux) nous restons en contact par mail. La veille d'un séance, nous fixons la salle et l'horaire, le travail à réaliser ayant était fixé à la fin de la séance précédente, et redéfinit en milieu par un réunion d'avancement.  

\chapter{Répartition des tâches dans le temps}

Nous avons mis en place un tableur Excel. Il nous permet de savoir ce qui à était la séance précédent et ce que nous devrons faire la séance suivante ; ainsi que le travail réalisé pendant la séance en vue des objectifs fixés. 

Ainsi chacun peut gérer au mieux l'ampleur de la tache à réaliser en fonction du temps qui lui est imparti.
 
Avant chaque séance, nous fessons un point pour fixer les objectifs du jour, de même en fin de séance pour fixer les objectifs de la prochaine séance. 
En milieu de chaque séance, nous nous réunissons pour voir l'avancement de chacun et redéfinir les taches si nécessaire. 

WBS et diagramme de Gantt
