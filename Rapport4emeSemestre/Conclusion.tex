\section*{Conclusion}
\addcontentsline{toc}{part}{Conclusion}

La première partie du projet HELP (partie IS - Ingénierie Système) nous a permis de comprendre les étapes qui mènent à l’aboutissement d’un projet. En effet, afin de se faire une idée de ce qui est réalisable, nous avons tout d’abord procédé à un état de l’art. Cette étude préalable nous a permis de comprendre que de nombreuses technologies existent déjà afin d’effectuer de l’eye tracking. A la fin de cette première phase, notre choix s’est orienté vers un système composé de deux caméras. Cependant lors de la réalisation de ce prototype, nous avons rencontré certaines difficultés qui nous ont emmenés à redéfinir et simplifier notre système. Nous nous sommes ainsi rendus compte de l’importance des phases d’expérimentation et de conception, notamment pour un projet d’une telle envergure. 
\bigbreak
De plus, si la première partie du projet est importante et que la partie sur l’état de l’art permet d’étudier un panel de solution envisageable à notre système, celui-ci n’a été réellement défini que lors de l'étape de conception. En effet, cette phase nous a permis de tester certaines idées, de faire des choix plus adaptés à nos compétences et d’améliorer, raffiner la partie IS du projet.
\bigbreak
Enfin, si nous avons revus nos exigences à la baisse durant ce projet, nous avons tout de même un démonstrateur fonctionnel et relativement robuste. De plus, la réalisation de celui-ci nous a plongés dans le domaine du gaze-tracking, encore peu exploré à ce jour. Cela nous a également permis de développer nos compétences en programmation, notamment en openCV et en C++, d'apprendre à travailler en équipe et de gagner en autonomie, grâce à la grande liberté de choix qui nous a été donnée par nos encadrants. Désormais, il serait intéressant d'embarquer notre système sur un casque ou des lunettes, solution envisagée mais que nous n'avons pas eu le temps de réaliser. Cela permettrait de résoudre les problèmes de mouvement de la tête.
