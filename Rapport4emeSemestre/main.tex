\documentclass[12pt]{report} %taille de la police par défaut
\input{packages}     % Ce fichier contient tous les packages nécessaires à la compilation
\makeindex           % donne l'ordre de créer l'index
\include{glossaire}  % ce fichier contient les entrées du glossaire
\makeglossaries      % donne l'ordre de créer le glossaire

\begin{document}
\def\chaptername{Partie}
\renewcommand{\contentsname}{Sommaire}                % des jolis noms pour la table des matières
\renewcommand{\bibname}{Références bibliographiques}  % des jolis noms pour les sections bibliographiques
\renewcommand{\glossaryname}{Glossaire}               % et glossaire
\setcounter{secnumdepth}{3}
\AddThinSpaceBeforeFootnotes % notes de bas de page
\FrenchFootnotes
\lstset{ %config texte brut (code)
language=c++,
basicstyle=\ttfamily\small, %
stepnumber=1
}

%----------------------------------------------------------------------------------------
%	 PAGE DE TITRE
%-----------------------------	-----------------------------------------------------------
\begingroup
\thispagestyle{empty}
\AddToShipoutPicture*{\put(6,5){\includegraphics[scale=1]{FondTitreSPID}}} % Image background
\begin{center}
\vspace*{4.5cm}
\includegraphics[scale=1]{Trait}
{\Huge \textsc{\textbf{Rapport écrit}}}\\
\vspace*{1cm}
{\Huge \textbf{Projet HELP}}\par % ACRONYME du projet
\vspace*{1cm}
{\huge Hope to Emulate the Life of Paralyzed people}\par % Intitulé du projet
\includegraphics[scale=1]{Trait}
\end{center}
%\vspace*{1cm}

\begin{center}
\large Rédigé par :\\
\end{center}

\begin{multicols}{2}
{\setlength{\baselineskip}{1.5\baselineskip}
{\large Semestre 3\\
Hussain Al Othman\\
Katleen Blanchet\\
Titouan Boulmier\\
Laure Dupasquier\\
Pierre Jacquot\\
Marie-Alice Schweitzer\\
\columnbreak\\
%\hspace*{1cm}
%\vspace*{1cm}
Semestre 4\\
Hussain Al Othman\\
Katleen Blanchet\\
Titouan Boulmier\\
Pierre Jacquot\\}
\par}
\end{multicols}

\begin{center}
\large
Sous la direction de :\\
Ali Mansour et Olivier Reynet
\end{center}

\vspace*{1.2cm}
\begin{center}
\large
22/05/2015
\end{center}


\endgroup


%----------------------------------------------------------------------------------------
%	COPYRIGHT PAGE
%----------------------------------------------------------------------------------------
%\newpage
%~\vfill
\thispagestyle{empty}

\noindent \bysa 2014-2015 Hussain Al Othman, Katleen Blanchet, Titouan Boulmier, Laure Dupasquier, Pierre Jacquot et Marie-Alice Schweitzer\\ % Copyright notice

%\noindent \textsc{Published by Publisher}\\ % Publisher

%\noindent \textsc{book-website.com}\\ % URL

\noindent Licensed under the Creative Commons Attribution-ShareAlike 4.0 International Public License.\\ % License information

\noindent \textit{Première impression, janvier 2015} % Printing/edition date

% REMERCIEMENTS
\newpage
\addcontentsline{toc}{part}{Remerciements}
\section*{Remerciements}

Nous tenons à remercier notre encadrant, M. Ali MANSOUR, pour ses conseils dans la réalisation physique du projet et sa disponibilité.
\bigbreak
Nous témoignons également nos remerciements à M. Olivier REYNET pour ses précisions sur l’ingénierie système et pour son accompagnement tout au long du projet.
\bigbreak
Nous remercions aussi les responsables de l’U.V. 3.4 pour leur présentation des techniques de gestion de projet et la mise en place des ateliers techniques.
\bigbreak
Enfin nous tenons à exprimer notre reconnaissance à l’ensemble du personnel de la médiathèque pour leur disponibilité et leurs éclaircissements sur la recherche de documents.

%----------------------------------------------------------------------------------------
%	SOMMAIRE
%----------------------------------------------------------------------------------------
\tableofcontents  % Imprime le sommaire
\cleardoublepage  % pour commencer sur une page impaire


%----------------------------------------------------------------------------------------
%	PART I 
%----------------------------------------------------------------------------------------
\addcontentsline{toc}{part}{Introduction au projet}
\section*{Introduction}


\subsection*{Contexte}

Dans le cadre de l'U.V. 3.4, notre équipe a choisi de développer le projet HELP (Hope to Emulate the Life of Paralyzed people) proposé par Mr Mansour. Ce projet a pour objectif la réalisation d'un système permettant de remplacer la souris d’ordinateur grâce aux mouvements des yeux. Cette étude semble très intéressante car elle demande une analyse et une compréhension de systèmes complexes d'eye tracking (oculométrie) déjà existants afin de mettre au point une version simplifiée et moins onéreuse. De plus, elle réunit différents aspects du travail d'ingénieur en informatique tels que le traitement de l'image, l'algorithmique, le travail en équipe,... Enfin, ce projet peut éventuellement mener à deux finalités différentes : d'abord, l'aide aux personnes tétraplégiques, qui, grâce à ce système, pourraient être moins dépendantes et retrouver un peu de liberté. Ensuite, ce projet pourrait aussi être utilisé afin d'aider les scientifiques à mettre au point un robot travaillant en zones hostiles qui puisse être contrôlé facilement grâce à la détection des mouvements de la tête et des yeux de l'opérateur.

\subsection*{Expression initiale du besoin}

Le but premier de ce projet était le développement d'un système permettant à une personne tétraplégique d'utiliser un ordinateur et surfer sur internet. Cependant, face à l'ampleur de la tâche, et suite à un entretien avec nos encadrants, nous avons décidé de commencer par développer un système permettant à une personne ordinaire de contrôler un ordinateur. Ainsi, le dispositif développé doit permettre à un utilisateur d'exécuter différentes applications sans avoir besoin de toucher une souris ou un clavier. L'usager doit pouvoir effectuer les opérations usuelles en bougeant et clignant des yeux.

\chapter{État de l'art}

\section{Présentation des technologies existantes}

\subsection{Systèmes avec contact}

Le suivi de l’œil (aussi appelé « eye tracking ») a de nombreuses applications, tels que le contrôle d’un système. Une pluralité de dispositifs a déjà été développée dans ce domaine et nous allons tâcher dans cette partie de vous présenter les procédés majeurs. Deux grandes catégories de systèmes existent : avec ou sans contact. Un système est dit sans contact lorsque celui-ci n’est pas attaché ou relié à l’utilisateur. Il a l’avantage d’être plus agréable est facile à utiliser. Le deuxième, avec contact, est monté sur l’usager (comme des lunettes par exemple). Il offre ainsi une plus grande mobilité. Dans notre cas, les deux méthodes sont à envisager.

\subsubsection{Les Tobii Glasses}

Les Tobbi Glasses \cite{tobiiglasses} sont des lunettes capables de filmer et d’enregistrer le mouvement des yeux. Elles peuvent ainsi savoir en temps réel ce que l’utilisateur fixe et voit.
Ces lunettes sont équipées de (cf. figure \ref{fig:TG}) : 
\begin{itemize}[label=\textbullet,font=\color{black}]
\item caméras qui filment directement l’œil
\item une caméra qui filme ce que voit l’utilisateur
\item plusieurs illuminateurs permettant d’éclairer l’œil
\item un capteur infrarouge
\end{itemize}

\begin{figure}[h]
  \centering
  \includegraphics[scale=0.4]{TobiiGlasses}
  \caption{Tobii Glasses}
  \label{fig:TG}
\end{figure}

Le capteur infrarouge permet au système de connaitre la position de la tête de l’utilisateur dans l’espace à l’aide d’émetteurs infrarouges balisant la zone qu'il regarde (cf. figure \ref{fig:Emetteur}).

\begin{figure}[H]
  \centering
  \includegraphics[scale=0.35]{BoitierTobii}
  \caption{Emetteur}
  \label{fig:Emetteur}
\end{figure}

Toutes les données sont enregistrées dans un boîtier relié aux lunettes (cf. figure \ref{fig:Boitier}). Elles peuvent ensuite être récupérées, traitées et analysées sur un ordinateur. La calibration est aussi effectuée via le boîtier.

\begin{figure}[H]
  \centering
  \includegraphics[scale=0.6]{BoitierTobii2}
  \caption{Boîtier}
  \label{fig:Boitier}
\end{figure}

D’un point de vue technique ces lunettes utilisent le Pupil Centre Corneal Reflection (PCCR, cf. figure \ref{fig:PCCR}). Cette méthode consiste à illuminer l’œil, créant ainsi un reflet sur la pupille et la cornée. Une caméra récupère ensuite une image de cette réflexion. Le vecteur formé par l’angle entre la cornée et le reflet lumineux sur la pupille est ensuite calculé. La direction correspond alors à la direction du regard de l’utilisateur.
Malheureusement l’algorithme de calcul n’est pas donné.

\begin{figure}[H]
  \centering
  \includegraphics[scale=0.8]{PCCR}
  \caption{Pupil Centre Corneal Reflection PCCR}
  \label{fig:PCCR}
\end{figure}

Le prix n’est pas communiqué sur le site internet, mais il semblerait qu’il soit d’environ 10 000 \euro{}.

\subsubsection{PUPIL}

\begin{figure}[h]
  \centering
  \includegraphics[scale=1]{PUPIL}
  \caption{PUPIL}
  \label{fig:PUPIL}
\end{figure}

PUPIL \cite{pupil} est un projet développé par trois étudiants du MIT. Tout comme les Tobbi Glasses il peut capter et enregistrer les mouvements effectués par l’œil de l’utilisateur.

Le principe de fonctionnement est basé sur deux caméras. La première permet d’enregistrer les mouvements de l’œil, et donc de retrouver par la suite la position de la pupille. La seconde filme ce qu’est censé voir l’utilisateur et permet de connaitre la direction du regard de l’utilisateur en utilisant les données de la première caméra.
Deux notions sont donc à distinguer : la position de la pupille et la direction du regard. La première nous aide à déduire la seconde. L’avantage de cette méthode est l’usage d'une caméra classique afin de détecter la direction du regard. Il n’est ici pas obligatoire d’employer la réflexion d’une lumière infra-rouge dans l’œil de l’utilisateur.

D’un point de vue matériel, les deux caméras utilisées sont très différentes.
Celle enregistrant le mouvement des yeux est une caméra basse résolution (640*480 à 30 fps) avec un filtre infrarouge (la  Microsoft LifeCam HD-6000 est recommandée). La caméra qui filme le monde alentour possède quant à elle une grande résolution (1920*1080, la Logitech HD 1080p Webcams est recommandée).
PUPIL peut être acheté pour la somme de 380 \euro{}.

\subsection{Systèmes sans contact}

\subsubsection{Tobii X2-30\&60}

\begin{figure}[h]
  \centering
  \includegraphics[scale=1]{TobiiX2}
  \caption{Tobii X2-30\&60}
  \label{fig:TobiiX2}
\end{figure}

Autre produit créé par Tobii, le X2 \cite{tobiieyetracker} rentre dans les systèmes d’oculométrie sans contact. Ne mesurant que 184 mm, il se branche directement en USB. Il est très facile d’utilisation et s'adapte à de nombreux supports (ordinateurs portables, tablettes, télévisions...).
Cet outil ne permet pas de contrôler le système sur lequel il est branché mais d’enregistrer ce que l’utilisateur regarde afin d’obtenir des statistiques.
La détection de la direction du regard se fait aussi à l’aide de la réflexion causée par des LEDs infrarouges sur la cornée du sujet. Combiné avec la localisation de la pupille, le système peut en déduire la direction du regard de l’utilisateur. Encore une fois l’algorithme n’est malheureusement jamais décrit. Cependant la précision de l’appareil permet une précision d’environ 0.34° sur la direction du regard du sujet. 

\subsubsection{Eye Charm}

\begin{figure}[h]
  \centering
  \includegraphics[scale=1]{EyeCharm}
  \caption{EyeCharm}
  \label{fig:EyeCharm}
\end{figure}

Créé par une société allemande (4tiito), EyeCharm \cite{eyecharm} est un adaptateur qui se clipse sur la Kinect et exploite sa caméra infrarouge pour suivre le mouvement des yeux. Un logiciel compatible avec Windows 7 \& 8 (NUIA) a été développé pour contrôler les principales fonctions d’un ordinateur grâce à ce système, comme le navigateur internet, les jeux vidéo et d’autres applications. L’utilisation d’EyeCharm ne nécessite aucun changement dans le code source des applications.
L’algorithme de suivi traite des images. La puissance de calcul nécessaire à son utilisation est ainsi assez importante (exemple : « son algorithme consomme 5 \% de la puissance d’un processeur Intel Core i5-3470 cadencé à 3,2 GHz »). Il est recommandé de détenir « un pc équipé d’un processeur AMD ou Intel multicœur et avec au moins 2 Go de mémoire vive ».
Le rôle d’EyeCharm est de projeter une lumière infrarouge sur le visage de l’utilisateur. Celle-ci est captée par la caméra de la Kinect (pour Xbox ou Windows, connectée en USB 2.0), ce qui lui permet de suivre le mouvement des yeux. Il est conseillé de se tenir à 75 cm de l’écran pour obtenir de bons résultats.

Le logiciel compte plusieurs extensions pour étendre les possibilités de contrôle par les yeux à :
\begin{itemize}[label=\textbullet,font=\color{black}]
\item plusieurs navigateurs internet (Chrome, Internet Explorer, Firefox)
\item Adobe Photoshop
\item la suite Office
\item les jeux World of Warcraft et Minecraft
\end{itemize}

Certaines fonctionnalités, comme le zoom ou le retour à la page précédente peuvent nécessiter une action supplémentaire, non réalisée par les yeux. Pour cela, il est possible d’appuyer sur une touche du clavier ou d’utiliser une commande vocale, prise en charge par la Kinect. 
Un kit de développement a été prévu pour que les utilisateurs puissent développer eux-mêmes des applications à l’aide de Qt Creator, Visual Studio, et les langages C, C++, C\# et Java.

\subsubsection{Robust Eye and Pupil Detection Method for Gaze Tracking \cite{gwon2013robust}}
\label{SystSC}

\begin{figure}[H]
  \centering
  \includegraphics[scale=0.6]{Experimental}
  \caption{Robust Eye and Pupil Detection Method for Gaze Tracking}
  \label{fig:Experimental}
\end{figure}

Ce système est intéressant car il n’est pas directement embarqué sur l’utilisateur. La détection des yeux se fait à l’aide de deux caméras. Ces dernières sont motorisées et peuvent donc tourner sur elle-même ou s’orienter vers le haut ou le bas. 
Une caméra dite wide-view filme l’utilisateur dans son intégralité. Dans les faits cette caméra permet de repérer son visage, puis la position de son œil grâce à deux algorithmes de reconnaissance faciale (l’Adaboost et le CAMShift). Une fois l’œil détecté, sa position est transmise à la seconde caméra dite narrow-view (d’une résolution de 1600x1200 réduite à du 240x320 pour améliorer le temps de calcul). Elle va pouvoir zoomer sur l’œil et ainsi obtenir un gros plan, tandis que la première ne sert qu’à repérer l’usager. Une fois l’œil dans le champ de vision de la caméra, la direction de celui-ci est calculée à l’aide du centre de la pupille et des réflexions spéculaires créées sur l’œil par quatre « near-infrared illuminators », eux même placés aux quatre coins de l’écran que regarde l’utilisateur. Ici encore l’algorithme de calcul reste flou.

D’un point de vue logiciel, cette méthode emploie du C++ et la librairie OpenCV. Tous les calculs sont effectués directement sur un ordinateur classique équipé d’un processeur Intel Core 2 Quad 2.3 GHz et de 4 GB.
N’étant qu’un sujet de thèse, ce montage n’est pas en vente.

\section{La détection de pupilles}

\subsection{L’œil}

Il est important de commencer par un rappel des caractéristiques biologiques de l’œil. Le lecteur acquerra certaines notions de bases qui lui permettront de mieux appréhender les différentes problématiques de la détection de pupilles.
La figure \ref{fig:PartiesOeil} nous présente les différentes parties de l’œil.

\begin{figure}[H]
  \centering
  \includegraphics[scale=1]{PartiesOeil}
  \caption{Schéma anatomique de l'œil humain}
  \label{fig:PartiesOeil}
\end{figure}

Dans le cadre de la détection de pupilles, nous nous intéresserons plus particulièrement à la partie visible de l’œil. Cette partie externe est composée de 3 zones :
\begin{itemize}[label=\textbullet,font=\color{black}]
\item La partie centrale : la pupille. C’est un orifice noir permettant de laisser passer la lumière.
\item La bande colorée entourant la pupille : l’iris. Il permet de plus ou moins dilater la pupille.
\item La zone blanche qui recouvre le reste de la partie visible : la sclérotique.
\end{itemize}

La figure \ref{fig:PartiesExternesOeil} présente ces 3 parties.
\begin{figure}[h]
  \centering
  \includegraphics[scale=1]{PartiesExternesOeil}
  \caption{Les différentes parties externes de l'œil humain}
  \label{fig:PartiesExternesOeil}
\end{figure}

Nous allons maintenant nous intéresser à la capture d’une image d’un œil, à son traitement puis à la détection de la pupille.

\subsection{Caractéristiques d’une image d’un œil acquise par caméra traditionnelle}

Il existe deux types de caméras pour acquérir l'image d’un œil : la caméra traditionnelle et la caméra infrarouge. Alors que la caméra traditionnelle permet de capturer le spectre de couleurs visibles par l’œil humain, la caméra infrarouge permet de capter les ondes infrarouges de la lumière naturelle (voir figure \ref{fig:SpectreLumiere}). Cependant, la lumière naturelle ne comprend que très peu de composantes infrarouges. Ainsi, il est souvent nécessaire de soumettre l’objet d’étude (l’œil pour notre projet) à une source de lumière infrarouge supplémentaire afin d’améliorer la qualité et la luminosité de l’image acquise. La caméra infrarouge permet d’éviter la majorité des problèmes de réflexion rencontrés par une caméra traditionnelle. Nous pouvons voir figure \ref{fig:CaptureOeilTrad} et figure \ref{fig:CaptureOeilInfra} les différents types d’image.

\begin{figure}[h]
  \centering
  \includegraphics[scale=1]{SpectreLumiere}
  \caption{Contenu spectral de la lumière}
  \label{fig:SpectreLumiere}
\end{figure}

\begin{figure}[!h] 
\centering
\subfigure[Traditionnelle]{
  \includegraphics[scale=0.7]{CaptureOeilTrad}
  \label{fig:CaptureOeilTrad}
}
\quad 
\subfigure[Infrarouge]{
  \includegraphics[scale=0.63]{CaptureOeilInfra}
  \label{fig:CaptureOeilInfra}
}
\caption{Capture d’œil à l’aide d’une caméra} 
\end{figure}

\subsection{Les différentes problématiques de la détection par caméra traditionnelle}

\subsubsection{Ombres}

La première difficulté rencontrée lors de la détection de pupilles par caméra traditionnelle est la présence de l'ombre des cils (voir figure \ref{fig:ZoneOmbre}). Cet ombrage est dû à un angle trop grand entre la source de lumière et l’axe de la pupille. Afin de réduire ce phénomène au maximum, une attention particulière doit être portée au positionnement et à l’axe de la source de lumière.

\begin{figure}[h]
  \centering
  \includegraphics[scale=1]{ZoneOmbre}
  \caption{Formation d’ombre dans une image d’un œil}
  \label{fig:ZoneOmbre}
\end{figure}

\subsubsection{Reflets}

La surface de l’œil étant lisse et recouverte par la cornée, structure transparente recouvrant et protégeant l’iris (figure \ref{fig:PartiesOeil}), les reflets émanant de l’œil sont considérables. 
L’intensité des reflets sont fonction de l’intensité lumineuse dégagée par la scène. Plus la scène sera lumineuse, plus elle sera reflétée dans l’œil.

Cependant, ces reflets sont gênants pour le traitement de l’image, et plus particulièrement pour la reconnaissance de la pupille. Par exemple, la figure \ref{fig:CaptureOeilTrad} présente un œil avec un reflet recouvrant une partie de la pupille et une partie de l’iris. Ainsi, la pupille n’est plus tout à fait ronde et cela peut poser problème lors de sa détection. 

Une attention particulière devra être portée aux éclairages de la pièce afin d'éviter les réflexions lumineuses ambiantes dues aux fenêtres, lampes et néons.. Cependant, il est important de noter qu’une réduction des sources lumineuses n’est pas une solution car l’image perdrait en clarté, en qualité, et son traitement n’en serait que plus difficile. La solution réside plus dans l’harmonisation de l’éclairage afin de ne pas créer de zone de réflexion.

Plusieurs approches sont possibles pour la résolution de ce problème de réflexion : soit par l’application d’opérateurs morphologiques\footnote{La morphologie mathématique est une théorie et technique mathématique et informatique d'analyse. Son développement est inspiré des problèmes de traitement d'images, domaine qui constitue son principal champ d'application. Elle fournit en particulier des outils de filtrage, segmentation, quantification et modélisation d'images.} en dilatant puis érodant la zone atteinte par la réflexion, soit en considérant qu’un reflet possède des caractéristiques contraires à celles de la pupille. Une source lumineuse étant très claire, on peut définir des bornes de niveau de gris à partir de l’histogramme de l’image et effectuer une segmentation (opérations de traitement d’images). Le seuillage de l’image originale de la pupille et l’image arborant les zones de réflexions seront combinés afin de former une image où les trous provoqués par les sources lumineuses seront remplis.

Il faut aussi noter que le port de verres correcteurs ou de lentilles accentue le phénomène de reflets.

\subsubsection{Pupille}

Il faut noter que si l’axe de la source de lumière correspond à celui des pupilles, il en résultera un effet d’yeux rouges qui peut être gênant pour le traitement de l’image et surtout la détection des pupilles. De plus, la taille des pupilles varie en fonction de l’intensité lumineuse présente. Il faudra donc veiller à ne pas imposer une source lumineuse trop importante afin d’avoir une taille de pupille suffisamment grande pour une bonne détection.

\subsection{Détection par éclairage infrarouge}
\label{EclInfra}

Si la caméra infrarouge permet d’éviter la majorité des problèmes de réflexion rencontrés par une caméra traditionnelle, sa mise en place est néanmoins plus complexe (utilisation d’émetteurs infrarouges afin d'éclairer l'oeil, utilisation de caméras infrarouges, ...) et plus cher. En effet, l'utilisation de caméras infrarouges multiplierait le coût du système par deux ou trois.

La détection de la pupille par illumination de l’œil avec des éclairages infrarouges se décline en deux grandes méthodes : la méthode dite de la pupille lumineuse (brigth pupil) et celle de la pupille noire (dark pupil). Suivant la position de l’éclairage infrarouge la pupille aura des propriétés optiques différentes. Si l’illumination infrarouge est coaxiale avec l’axe optique de la caméra alors la pupille apparaitra totalement blanche (bright pupil). Au contraire si la source lumineuse est décalée (toujours par rapport à l’axe optique), alors la pupille apparaitra noire (dark pupil) et l’on pourra voir distinctement les reflets de la source lumineuse infrarouge sur la cornée du sujet.

\begin{figure}[H]
  \centering
  \includegraphics[scale=1]{BrightPupil}
  \caption{Effet black et bright pupil}
  \label{fig:BrightPupil}
\end{figure}

\subsection{Traitement d’images}

\subsubsection{Segmentation}

La segmentation (aussi appelée Clustering) \cite{d2004etude} est une étape de base du traitement d’images qui a pour but de séparer différentes zones homogènes d’une image en groupes (clusters) dont les membres ont en commun diverses propriétés (intensité, couleur, texture...). En d’autres mots, cette opération rassemble des pixels entre eux suivant des critères prédéfinis. Les pixels sont ainsi regroupés en régions, qui constituent un pavage ou une partition de l'image. Il peut s'agir par exemple de séparer les objets du fond. On peut regrouper les différentes méthodes de segmentation en deux catégories : la segmentation non supervisée, qui vise à séparer automatiquement l’image en différents clusters naturels (sans aucune connaissance préalable des classes) ; et la segmentation supervisée, qui s’opère à partir de la connaissance de chacune des classes définie. Concernant les méthodes de segmentation non supervisée, nous nous intéresserons à deux méthodes : le Fuzzy C-means et le k-means. 

\subsubsection*{Description des algorithmes}
Les principales étapes d'un algorithme de classification sont :
\begin{enumerate}
\item La fixation arbitraire d’une matrice des classes.
\item Le calcul des centroïdes des classes.
\item Le réajustement de la matrice d’appartenance suivant la position des centroïdes.
\item Calcul du critère de minimisation et retour à l’étape 2 s’il y a non convergence de critère.
\end{enumerate}
\paragraph{}

Fuzzy C-Means (FCM) est un algorithme de classification non supervisée floue. Il introduit la notion d’ensemble flou dans la définition des classes : chaque point (pixel) dans l’ensemble des données appartient à chaque cluster avec un certain degré d’appartenance, et tous les clusters sont caractérisés par leur centre de gravité. Ainsi, il permet d’obtenir une partition floue de l’image en donnant à chaque pixel un degré d’appartenance compris entre 0 et 1 à un cluster donné. Le cluster auquel est associé un pixel est celui dont le degré d’appartenance est le plus élevé.

La classification floue des objets est décrite par une matrice floue à n lignes et c colonnes dans laquelle n est le nombre d’objets de données et c est le nombre de grappes. $\mu$ij est l’élément de la i-ème ligne et la colonne j. $\mu$, représente le degré de fonctiond’appartenance de l’i-ème objet avec le pôle j.
Les étapes de l’algorithme FCM sont présentées figure \ref{fig:ALGOFCM}:

\begin{figure}[H]
  \centering
  \includegraphics[scale=0.6]{AlgoFCM}
  \caption{Pseudo Code de Fuzzy C-Means}
  \label{fig:ALGOFCM}
\end{figure}

Le FCM est un algorithme très puissant, mais son inconvénient principal réside dans l’initialisation des centres.

\paragraph{}
L’algorithme k-means est l’algorithme de Clustering le plus connu et le plus utilisé (simple à mettre en œuvre). Il permet de partitionner les pixels d’une image en K clusters. L’algorithme k-means ne crée qu’un seul niveau de clusters. Il renvoie une partition des données, dans laquelle les objets à l’intérieur de chaque cluster sont aussi proches que possible les uns des autres et aussi loin que possible des objets des autres clusters. Chaque cluster de la partition est défini par ses objets et son centroïde. Le k-means est un algorithme itératif qui minimise la somme des distances entre chaque objet et le centroïde de son cluster. La position initiale des centroïdes conditionne le résultat final, de sorte qu'ils doivent être initialement placés le plus loin possible les uns des autres de façon à optimiser l’algorithme. K-means réitère ses opérations jusqu’à ce que la somme ne puisse plus diminuer. Le résultat est un ensemble de clusters compacts et clairement séparés, à condition d’avoir choisi la bonne valeur K du nombre de clusters. Les principales étapes de l’algorithme k-means sont :
\begin{enumerate}
\item Choix aléatoire de la position initiale des K clusters.
\item (Ré-)Affecter les objets à un cluster suivant un critère de minimisation des distances (généralement selon une mesure de distance euclidienne).
\item Une fois tous les objets placés, recalculer les K centroïdes.
\item Réitérer les étapes 2 et 3 jusqu’à ce que plus aucune réaffectation ne soit faite.
\end{enumerate}
\paragraph{}
Statistics Toolbox implémentée sous MatLab 7 contient la fonction kmeans.m, facile à prendre en main et bien documentée et Fuzzy Logic Toolbox contient fcm.m.

\subsubsection*{Comparaison}
Il apparait que ces deux algorithmes sont assez efficaces pour des images couleurs et permettent une bonne segmentation. Cependant, lors de la présence de défauts (reflets, ombres...), la segmentation parait correcte (elle ne permet pas une bonne détection des caractères par Optical Character Recognition OCR, ce qui ne nous intéresse pas). Enfin, il apparait que ces algorithmes ne sont pas adaptés à des images contenant un grand nombre d’objets (au niveau du temps de calcul). La méthode FCM semble plus efficace pour la haute résolution, et au contraire, k-means convient plus aux images à faible résolution.

\subsubsection{Traitement de l’image pour la détection de pupilles}

La détection des pupilles est composée de plusieurs étapes et peut être codée à l’aide de plusieurs méthodes différentes en fonction de la complexité choisie. Certaines méthodes sont plus complexes et meilleures mais plus longues à réaliser (au niveau du temps de calcul). Pour notre projet, nous avons besoin de faire du traitement d’images en temps réel donc nous devrons faire attention à la complexité de notre algorithme. Les différentes étapes de la détection de la pupille sont : 
\begin{enumerate}
\item Filtrage du bruit (filtre médian...).
\item Prétraitement (égalisation histogramme + seuillage bas pour supprimer le reflet).
\item Extraction de contour (filtre variance, ...).
\item Recherche des cercles/ellipses (Hough).
\item Discrimination du cercle de la pupille (couleur, position, surface, ...).
\end{enumerate}
\paragraph{}
Une première méthode de détection de la pupille, développée par Jérôme Schmaltz de l’Ecole de Technologie Supérieure de Montréal, repose sur le principe décrit figure \ref{fig:TraitementImage}.


\subsubsection*{Suppression de surplus de contours, atténuation du bruit}
\label{Filtre}

Tout d’abord, il est nécessaire de réduire le bruit occasionné par la caméra, le bruit engendré par le transfert des données et leur compression. Différents filtres \cite{bergounioux2010quelques} peuvent être utilisés, même si leur application altérera les contours de l’image. 

\begin{figure}[H]
  \centering
  \includegraphics[scale=1]{TraitementImage}
  \caption{Différentes étapes de traitement de l’image afin de détecter une pupille dans une image.}
  \label{fig:TraitementImage}
\end{figure}

\subsubsection*{Filtre Moyenneur}

Ce filtre lisseur part du principe que la valeur d'un pixel est relativement similaire à son voisinage. Il fait donc en sorte que chaque pixel soit remplacé par la moyenne pondérée de ses voisins. Si on applique un filtre moyenneur de taille $\lambda$=3, cela signifie qu'on additionne la valeur de tous les pixels du voisinage du pixel traité. On obtient ainsi la matrice de convolution suivante :
$$h = \frac{1}{9}
\begin{bmatrix}
	1 & 1 & 1 \\
	1 & 1 & 1 \\
	1 & 1 & 1 \\
\end{bmatrix}  $$

\paragraph{}
h s’appelle le masque de convolution. La somme des coefficients du masque valant 1, le lissage préservera toute zone de l’image où le niveau de gris est constant. Ce filtre peut engendrer des phénomènes de fausses couleurs, contrairement au filtre médian car son efficacité est moindre lorsque les objets présents dans l'image sont de faible dimension par rapport aux dégradations. Ce filtre est isotrope. 

Une amélioration du filtre moyenneur consiste à jouer sur les valeurs des coefficients du masque :
$$h = \frac{1}{10}
\begin{bmatrix}
	1 & 1 & 1 \\
	1 & 2 & 1 \\
	1 & 1 & 1 \\
\end{bmatrix}  $$

\subsubsection*{Coupe Médiane}

Une coupe médiane permet d’atténuer le bruit d’une image. Le principe est de prendre dans le voisinage la valeur la moins extrême. Pour cela, on crée une liste des valeurs du voisinage, puis on trie cette liste et on prend la valeur qui se trouve au milieu de la liste. Cette valeur "médiane" est la plus éloignée des deux extrêmes.

\subsubsection*{Diffusion}

Le principe est d’atténuer les différences d'intensité entre le pixel central et ses voisins. Pour chaque voisin, on calcule la différence d'intensité avec le pixel central. Plus la différence est faible, plus elle est propagée vers le pixel central. Cela permet d'uniformiser les zones d'intensité proche et de conserver les forts contrastes (et donc les contours).

\subsubsection*{Filtre Gaussien}
\label{FiltreGaussien}

Le filtre Gaussien est un filtre isotrope spécial avec des propriétés mathématiques bien précises. $\sigma$ caractérise l'écart type soit la largeur du filtre : autrement dit la largeur du filtre en partant du point central est égale à 3$\sigma$ arrondi à l'entier supérieur. En deux dimensions, le filtre de Gauss est le produit de deux fonctions gaussiennes, une pour chaque direction :

$$g(x,y) = \frac{1}{2\pi\sigma^2}\cdot \mathrm{e}^{-\frac{x^2+y^2}{2\sigma^2}}$$

L’effet de ce filtre sur l’image est assez similaire au filtre moyenneur, mais la moyenne est pondérée : les pixels près du centre ont un poids plus important que les autres. En général, un filtre Gaussien avec $\sigma$<1 est utilisé pour réduire le bruit, et si $\sigma$>1 c’est dans le but de flouter volontairement l'image. Il faut noter que plus $\sigma$ est grand, plus la cloche Gaussienne est large, et plus le flou appliqué à l’image sera marqué.

\subsubsection{Conversion en niveau de gris}

La seconde étape du prétraitement consiste à convertir l’image RGB en niveau de gris. Cette conversion est nécessaire puisque les différents algorithmes se basent sur des images en niveau de gris. On pourrait d’abord penser que le niveau de gris se calcule comme la somme des 3 pixels divisée par 3 : $Gris = \frac{R + G + B}{3}$

Cependant, d’après les recommandations de la commission internationale de l’éclairage, il devrait plutôt être de la forme : $Gris = 0.299R + 0.587G + 0.114B$
Nous pouvons voir figure \ref{fig:Conversion} le rendu de cette conversion.

\begin{figure}[h]
  \centering
  \includegraphics[scale=1]{Conversion}
  \caption{Conversion en niveau de gris}
  \label{fig:Conversion}
\end{figure}

\subsubsection{Détection des contours}

\subsubsection*{Filtre de Canny}

Il existe différentes méthodes de détection des contours. Des filtres peuvent être utilisés (voir paragraphe \ref{Filtre}). Le choix du filtre doit prendre en compte qu’une réduction trop importante du bruit masque certains contours, mais qu’une réduction trop faible peut laisser apparaitre trop de contours non significatifs. Nous nous intéresserons ici au meilleur filtre pour la détection de contours : le filtre de Canny. Il est considéré comme tel car il offre un très bon compromis entre la réduction du bruit et la localisation des contours. Les différentes étapes du filtre de Canny sont :
\begin{itemize}[label=\textbullet,font=\color{black}]
\item Réduction du bruit grâce à la convolution d’un filtre passe-bas Gaussien, ce qui permet de tamiser les contours.
\item Eclaircissement des contours grâce à l’utilisation des valeurs des amplitudes des gradients ainsi que leurs directions. Pour cela, nous utilisons les formules \eqref{eq:g} et \eqref{eq:theta}.
\end{itemize}

\begin{equation}
g = \sqrt{g_x^2 + g_y^2}
\label{eq:g}
\end{equation}
\begin{equation}
\theta = \arctan{\frac{g_y}{g_x}}
\label{eq:theta}
\end{equation}

Nous pouvons voir que l’amplitude est proportionnelle à $g_x$ et $g_y$, ce qui montre qu’elle mesure la force d’un contour indépendamment de sa direction. 
\begin{itemize}[label=\textbullet,font=\color{black}]
\item Suppression des non maximaux, cela permet de réduire la largeur des arrêtes détectées à un pixel. La figure \ref{fig:Algo} décrit un algorithme permettant de supprimer ces non maximaux.
\begin{figure}[h]
  \centering
  \includegraphics[scale=0.9]{Algo}
  \caption{Algorithme permettant de supprimer des non maximaux}
  \label{fig:Algo}
\end{figure}
\item Localisation : elle repose sur l’identification des contours significatifs à partir des amplitudes des gradients précédemment calculés. 
\end{itemize}

La méthode triviale consiste à appliquer un seuillage aux gradients calculés. Cependant, l’application d’un seuil trop faible implique la prise en compte de tous les contours, y compris les contours non significatifs (prise en compte des gradients maximaux engendrés par le bruit). De plus, l’application d’un seuil trop élevé engendre une fragmentation des chaines de pixels qui forment des contours significatifs de l’image. Ainsi, le seuil par hystérésis offre une bonne solution à ce problème. 

La figure \ref{fig:DetectionContours} illustre l’idée de détection des contours grâce au filtre de Canny.
\begin{figure}[H]
  \centering
  \includegraphics[scale=0.9]{DetectionContours}
  \caption{Détection des contours grâce au filtre de Canny}
  \label{fig:DetectionContours}
\end{figure}

\subsubsection*{Filtre de Sobel}

L'opérateur calcule le gradient de l'intensité de chaque pixel. Celui-ci indique la direction de la plus forte variation (du clair au sombre), ainsi que le taux de changement dans cette direction. On connaît alors les points de changement soudain de luminosité, correspondant probablement à des bords, ainsi que l'orientation de ces bords.
L'opérateur utilise des matrices de convolution. La matrice (généralement de taille 3×3) subit une convolution avec l'image pour calculer des approximations des dérivées horizontale et verticale. Soit $A$ l'image source, $G_x$ et $G_y$ deux images qui en chaque point contiennent des approximations respectivement de la dérivée horizontale et verticale de chaque point. Ces images sont calculées comme suit:

$$G_x = 
\begin{bmatrix}
	1 & 0 & -1 \\
	2 & 0 & -2 \\
	1 & 0 & -1 \\
\end{bmatrix}
* A \text{ et } G_y =
\begin{bmatrix}
	1 & 2 & 1 \\
	0 & 0 & 0 \\
	-1 & -2 & -1 \\
\end{bmatrix}
* A
$$

En chaque point, les approximations des gradients horizontaux et verticaux peuvent être combinées comme suit pour obtenir une approximation de la norme du gradient : $G = \sqrt{G_x^2 + G_y^2}$

On peut également calculer la direction du gradient comme suit : $\Theta = atan2(G_y, G_x)$, et créer une matrice des directions.

\begin{figure}[h]
  \centering
  \includegraphics[scale=1]{FiltreSobel}
  \caption{Déterminer les directions des gradients à partir de l’application du filtre de Sobel}
  \label{fig:FiltreSobel}
\end{figure}

Enfin, d’autres filtres tels que le filtre de Kirsch, le filtre MDIF, le filtre de Prewitt ou encore celui de Roberts, permettent aussi la détection de contours, mais nous ne les détaillerons pas ici.

\subsubsection{Détection des centres}

Le principe de la détection des centres se base sur les deux étapes précédentes. Pour chaque point des contours de l’image de Canny, on trace une droite en fonction de la direction calculée par le biais de l’application des filtres de Sobel. Ainsi, par l’application de la formule de la droite $y = m \cdot x + b$, on trace une droite dans l’accumulateur en fonction des points de contours de l’image de Canny tels qu’exposés dans la figure \ref{fig:Contour}.

\begin{figure}[h]
  \centering
  \includegraphics[scale=1]{Contour}
  \caption{Traçage d’une droite en fonction d’un contour}
  \label{fig:Contour}
\end{figure}

Tracer une droite dans un accumulateur revient à incrémenter chaque cellule d’une matrice aux endroits où elle passe (voir figure \ref{fig:Accumulateur}).

\begin{figure}[h]
  \centering
  \includegraphics[scale=1]{Accumulateur}
  \caption{Traçage d’une droite dans l’accumulateur}
  \label{fig:Accumulateur}
\end{figure}

Ainsi, suivant ce principe, les valeurs maximales de la matrice indiqueront les différents centres potentiels. La figure \ref{fig:MethodeCanny} synthétise la méthode énoncée. 

\begin{figure}[h]
  \centering
  \includegraphics[scale=0.9]{MethodeCanny}
  \caption{Traçage des lignes dans l’accumulateur en fonction de l’image de Canny et de la matrice de directions}
  \label{fig:MethodeCanny}
\end{figure}

\subsubsection{Seuillage}

Le seuillage permet de désencombrer la matrice accumulateur des informations superflues. Nous recherchons les centres potentiels et ainsi nous n’avons pas besoin de toutes les valeurs de l’accumulateur. En effet, il suffit d’appliquer un seuil et de garder les pixels ayant une valeur supérieure à celui-ci (les autres pixels seront mis à zéro) car ils représenteront les différents centres potentiels. 

Pour ce faire, nous pouvons par exemple utiliser un seuil du type :

$\text{valeur pixel actuel} > 0.5 \cdot max(\text{valeur pixel image})$, qui permet de garder les pixels ayant au moins la moitié de la valeur maximale d’une cellule de l’accumulateur.

Nous pouvons voir figure \ref{fig:SeuilAccumu} le résultat du seuillage.

\begin{figure}[h]
  \centering
  \includegraphics[scale=1]{SeuilAccumu}
  \caption{Application d’un seuil dans l’accumulateur}
  \label{fig:SeuilAccumu}
\end{figure}

\subsubsection{Isoler les centres potentiels}

L’application d’une convolution d’un chapeau mexicain (voir formule \eqref{eq:RepLaplacienGaussien} et figure \ref{fig:LaplacienGaussien}) sur l’accumulateur permet d’isoler les points les plus au centre. Un chapeau mexicain est la combinaison d’un filtre Gaussien (voir paragraphe \ref{FiltreGaussien}) avec un filtre Laplacien.

\begin{equation}
\nabla^2G_\sigma(r) = \frac{-1}{2\pi\sigma^4}(2-\frac{r^2}{\sigma^2}) \mathrm{e}^{\frac{-r^2}{2\sigma^2}}
\label{eq:RepLaplacienGaussien}
\end{equation}

\begin{figure}[H]
  \centering
  \includegraphics[scale=1]{LaplacienGaussien}
  \caption{Représentation graphique d’un Laplacien de Gaussien}
  \label{fig:LaplacienGaussien}
\end{figure}

Ainsi, l’application d’un chapeau mexicain sur l’accumulateur permet de conserver les points les plus à même d’être les centres de cercles potentiels en décrémentant les valeurs des cellules entourant les centres. La figure \ref{fig:ChapeauMexicain} présente le résultat de cet isolement des centres.

\begin{figure}[H]
  \centering
  \includegraphics[scale=1]{ChapeauMexicain}
  \caption{Application d’un chapeau mexicain dans l’accumulateur}
  \label{fig:ChapeauMexicain}
\end{figure}

\subsubsection{Recherche des cercles}

La recherche des cercles s’appuie sur la création d’accumulateurs et la transformée de Hough. En effet, pour un cercle de rayon inconnu, nous utilisons des accumulateurs en 3 dimensions : les deux premières servent à ranger les coordonnées x et y du centre du cercle et la troisième permet de ranger le rayon r du cercle. 

\begin{enumerate}
\item On initialise les accumulateurs.
\item A partir de l’image des contours de Canny, on trace littéralement des cercles dans un des accumulateurs en faisant coïncider les centres des cercles avec les coordonnées du pixel de contour pour un rayon donné. La méthode de Bresenham pourra être employée afin de tracer plus efficacement les cercles dans l’accumulateur de Hough.
\item Comme on peut le voir figure \ref{fig:CerclesContours}, plusieurs cercles sont tracés (dans un accumulateur de rayon r) en suivant les contours de l’image de Canny. Nous pouvons voir que les cellules contenant les valeurs maximales coïncident avec des centres de cercles potentiels.

\begin{figure}[H]
  \centering
  \includegraphics[scale=1]{CerclesContours}
  \caption{Traçage des cercles avec l’image des contours}
  \label{fig:CerclesContours}
\end{figure}

\item Ensuite, on peut diviser la zone d’intérêt de l’image en deux parties : partie droite et partie gauche du visage. Pour chacune de ces parties, on applique un seuil aux accumulateurs d’Hough permettant de conserver les centres, les coordonnées et rayons de cercles potentiels et on vérifie qu’ils font bien partie de l’accumulateur des cercles potentiels de l’étape de l’isolement des centres potentiels.
\item Pour finir, une moyenne des positions et des rayons des cercles restants permet une identification des cercles entourant l’iris et la pupille (à gauche et à droite). Il ne reste plus qu’à conserver le cercle ayant le rayon le plus petit, correspondant à la pupille.
\end{enumerate}

La figure \ref{fig:OpPupilles} résume cette démarche.

\begin{figure}[H]
  \centering
  \includegraphics[scale=0.7]{OpPupilles}
  \caption{Les différentes opérations permettant de localiser les pupilles}
  \label{fig:OpPupilles}
\end{figure}

\section{Méthode de détection de la direction du regard}
\label{MethodeDirectionRegard}

La direction du regard d’une personne peut être obtenue à l’aide de deux paramètres : 
L’orientation du visage et l’orientation oculaire. La détection du visage permet de connaitre l’orientation globale de l’utilisateur et de vérifier que ses deux yeux sont bien en face de la caméra. Suivant l’orientation de la tête de l’utilisateur, une approximation de la direction du regard peut aussi être réalisée.
La détection effectuée à l’aide de l’orientation oculaire approche la direction du regard à l’aide de propriétés géométriques et physiques de l’iris et de la pupille. Le problème principal réside dans le fait que cette étude est une étude locale, c’est-à-dire que la caméra qui filme l’œil de l’utilisateur doit avoir un fort niveau de zoom, obligeant le sujet à rester immobile afin de ne pas sortir du champ de la caméra.

Les reflets détectés dans l’œil du sujet par la détection de pupille avec infrarouge (voir paragraphe \ref{EclInfra}) vont permettre de déterminer la direction du regard de l’utilisateur. En supposant la tête de l’utilisateur immobile, et en prenant comme référence le reflet formé par la lumière infrarouge sur la cornée, on peut déterminer la direction du regard en se servant du vecteur crée entre le reflet et le centre de la pupille (ou l’iris) (voir figure ci-dessous). Cependant cette méthode demande certains prérequis et notamment une calibration et ce avec chaque nouvel utilisateur. Pour effectuer cette calibration, il est demandé au sujet de fixer quatre points (minimum), situés sur l’écran, afin d’obtenir des valeurs étalons (souvent les coins de l’écran), qui permettront de déduire la direction du regard par la suite.

\begin{figure}[H]
  \centering
  \includegraphics[scale=1]{GazeDetection}
  \caption{Tracé du vecteur entre la réflexion infrarouge dans la cornée et la pupille}
  \label{fig:GazeDetection}
\end{figure}

Cette méthode présente tout de même deux défauts majeurs : l’impossibilité pour l’utilisateur de bouger la tête ou le corps, sous risque de devoir recommencer la calibration, et l’obligation de re-calibrer le système à chaque changement d’utilisateur.

\section{Solutions techniques et méthodes envisagées} 

\subsection{Au préalable, suite à l'état de l'art}

Suite à l’état de l’art, la méthode d’eye tracking sans contact a été retenue (pas de système embarqué). Elle est similaire à celle de la partie \ref{SystSC} et semble plus facilement réalisable qu'un système utilisant des lunettes avec caméras embarquées. Ainsi, dans un premier temps, cette technique sera développée afin d’obtenir un système qui fonctionne correctement. Puis, le système pourra être perfectionné et, s’il reste du temps sur les délais fixés, la méthode utilisant des lunettes sera approfondie. 

Pour l'approche sans contact, deux caméras seront utilisées. La première, la caméra dite "grand angle", permettra de filmer la scène dans son ensemble et de localiser le visage ainsi que certains de ses éléments morphologiques tels que les oreilles et la bouche. Elle transmettra ensuite les coordonnées spatiales du visage à la seconde caméra "petit angle". Cette dernière zoomera sur le visage (grâce aux coordonnées fournies) et effectuera un tracking des pupilles.  

La détection de la pupille s’effectuera à l’aide de la méthode de la bright et de la black pupille. Les différentes étapes vues dans l’état de l’art seront respectées et des tests seront menés afin de déterminer la meilleure solution pour réaliser chaque étape.
 
Le calcul de la direction du regard s’appuiera sur la méthode vue dans la partie \ref{MethodeDirectionRegard}. Des LEDs infrarouges, placées devant l’utilisateur, illumineront sa cornée, créant une réflexion spéculaire exploitable pour estimer la direction du regard. La caméra "petit angle" devra ainsi être munie d’un filtre infrarouge. Ce filtre pourra être réalisé à l’aide d’une pellicule photo placée devant l’objectif de la caméra. 

\subsection{Méthodes choisies lors de la réalisation} 
Devant notre effectif réduit lors de la deuxième phase du projet, nous avons décidé de nous concentrer sur l'utilisation d'une seule caméra combinant "grand angle" et "petit angle". En effet, cela nous a permis d'éliminer tout problème de calibrage entre les deux caméras. De plus, nous avons constaté que la résolution de notre webcam était suffisamment élevée pour la détection du centre des pupilles. Ainsi, puisque l'infrarouge, qui permet certes une détection plus précise, est inconfortable et moins robuste à la lumière du jour, nous avons décidé d'abandonner cette technologie dans un premier temps.

%----------------------------------------------------------------------------------------
%	PART II 
%----------------------------------------------------------------------------------------
\chapter{Dossier fonctionnel}
\section{Ingénierie des exigences}
\subsection{Approche Top-Down}
\label{sec:top-down}

Pour notre approche Top-Down, nous sommes revenus à la demande initiale du projet : remplacer la souris d'un ordinateur par le mouvement oculaire de l'utilisateur. La fonction principale du système est donc apparue clairement : permettre à l'utilisateur d'interagir avec une interface via ses yeux. 

\begin{figure}[h]
  \centering
  \includegraphics[scale=1]{BeteACornes}
  \caption{Bête à cornes}
  \label{fig:bac}
\end{figure}

\begin{figure}[H]
  \centering
  \includegraphics[scale=0.9]{PieuvreV2}
  \caption{Diagramme pieuvre}
  \label{fig:pieuvre}
\end{figure}

\subsection{Approche Bottom-Up}
\definecolor{sable}{RGB}{238,236,225}
Nous avons défini les différentes exigences auxquelles le système devra répondre, indépendamment de nos choix de réalisation technique. Nous les avons alors rassemblées par groupement logique, ce qui nous permettra de définir nos fonctions principales. Les lignes en {\color{sable}{\rule{0.5cm}{0.25cm}}} sont uniquement valables pour la méthode embarquant un système sur l'utilisateur. Elles sont donc à ignorer dans un premier temps puisque nous avons choisi une approche différente. 

\begin{figure}[H]
  \centering
  \includegraphics[width=\textwidth]{CahierdesExigencesV3}
  \caption{Cahier des exigences}
  \label{fig:exigences}
\end{figure}

\subsection{Fonctions principales du système}

Dans le cadre de ce projet, nous cherchons donc à remplacer la souris d'un ordinateur par un système qui suit le mouvement des yeux de l'utilisateur. Pour ce faire, nous avons mis en évidence des groupements logiques d'exigence. Tout d'abord, le système doit acquérir les données nécessaires à la détection du mouvement de l'œil. Ces données devront alors être traitées par l'ordinateur. Ce dernier doit alors interpréter ces données pour en déduire l'action à exécuter. De ces nouvelles informations, l'ordinateur doit pouvoir effectuer l'action que l'utilisateur veut effectuer sur l'IHM, la mettre en place, et montrer que ces modifications ont été exécutées. 

\begin{itemize}[label=\textbullet,font=\color{black}]
\item FP1 : Acquérir les données nécessaire à la détection du mouvement de l'œil 
\item\colorbox{sable}{FP2 : Transférer les données enregistrées}
\item FP3 : Extraire les repères anatomiques
\item FP4 : Interpréter les données pour déduire l'action à exécuter 
\item FP5 : Agir sur l'IHM 
\end{itemize}

\section{Spécification fonctionnelle  3 axes}

\subsection{Raffinement FAST}

\begin{figure}[H]
  \centering
  \includegraphics[scale=0.70]{FASTV2}
  \caption{FAST raffiné}
  \label{fig:FAST}
\end{figure}

Le cahier des exigences précédant nous a permis de définir les fonctions principales du système. Nous avons alors cherché à les raffiner par la méthode FAST (cf. figure \ref{fig:FAST}) pour obtenir des fonctions plus précises auxquelles nous pourront apporter des solutions techniques propres à chacune. Nous pouvons ainsi entrevoir l'architecture fonctionnelle.

\subsection{Spécification des données}

La spécification du flux de données va permettre de comprendre les interactions et les échanges entre les différentes parties de notre système. Spécifier ainsi les données nous a permis de mieux comprendre les interactions entre chaque sous-partie du système. Il est ainsi clair que les données que nous allons principalement traiter et transférer seront des flux vidéos et qu’il sera donc important d’optimiser au mieux la rapidité de ces échanges.
Au travers de la figure \ref{fig:fluxDonnees}, nous pouvons aussi voir l’ordre avec lequel les sous-systèmes vont rentrer en jeu. Ainsi la Camera Narrow View sera dépendante du flux envoyé par la Camera Wide View. Un parallélisme de traitement des données est tout de même envisageable sur l'ordinateur pour gérer le flux vidéo de deux caméras.

\begin{figure}[h]
  \centering
  \includegraphics[scale=0.6]{FluxDonnees}
  \caption{Flux de données}
  \label{fig:fluxDonnees}
\end{figure}

\subsection{Spécification des comportements}

Le diagramme de séquence de la figure \ref{fig:comportementGlobal} illustre le comportement global du système. 

\begin{figure}[H]
  \centering
  \includegraphics[scale=0.6]{comportementGlobal}
  \caption{Diagramme des spécifications du comportement global}
  \label{fig:comportementGlobal}
\end{figure}

\section{Architecture fonctionnelle}

Tout le travail réalisé au préalable sur la spécification fonctionnelle 3 axes permet finalement de proposer une architecture fonctionnelle pour notre système (cf. figure \ref{fig:archiFonctionnelle}).
 
\begin{figure}[H]
  \centering
  \includegraphics[scale=1.2]{ArchitectureFonctionnelle}
  \caption{Architecture Fonctionnelle}
  \label{fig:archiFonctionnelle}
\end{figure}



%----------------------------------------------------------------------------------------
%	PART III 
%----------------------------------------------------------------------------------------
\chapter{Implémentation}

\section{Architecture physique et interfaces}

Grâce à l'architecture physique (cf figure \ref{fig:archiPhysique}), nous avons pu déduire que notre projet se décomposait en trois sous-systèmes : les interfaces d'acquisition, logicielle et graphique.

\begin{figure}[H]
  \centering
  \includegraphics[scale=1.2]{architecturePhysique}
  \caption{Architecture Physique}
  \label{fig:archiPhysique}
\end{figure}

\subsection{Interface d'acquisition}
Cette interface est constituée principalement de la caméra et de la mentonnière. Cette dernière est utilisée pour éviter les mouvements de la tête, non traités par le programme. L'ordinateur occupe également une place secondaire, puisqu'il permet de vérifier que l'utilisateur est convenablement installé.

\subsection{Interface logicielle}
L'interface logicielle s'appuie sur le programme contenant des algorithmes développés en C++ avec OpenCV. Ils ont pour but de traiter les images transmises par la webcam afin de détecter le visage, puis les yeux et enfin les pupilles et les clignements pour déduire l'action que l'utilisateur souhaite effectuer. Que ce soit pour le clic ou le mouvement du curseur de la souris, une interaction avec l'OS est nécessaire. Le programme doit donc être adapté suivant le système d'exploitation utilisé.

\subsection{Interface graphique}
Enfin, le rôle de l'interface graphique est de proposer à l'utilisateur un choix de boutons qui lancent des applications. Cette dernière est développée en C++ à l'aide de Qt Creator et ne peut être utilisée que sur Windows car il faudrait changer toutes les commandes d'ouverture d'applications pour exécuter le programme sur un autre système d'exploitation.

\section{WBS}
La structure de découpage du projet découle de l'architecture physique. La figure \ref{fig:WBS} illustre le WBS ainsi établit. 

\begin{figure}[H]
  \centering
  \includegraphics[scale=0.8]{WBS}
  \caption{WBS}
  \label{fig:WBS}
\end{figure}
\section{Plan de validation fonctionnelle}

Afin de vérifier le bon fonctionnement de notre programme, nous avons établi une stratégie de validation système. L'annexe \ref{APV} spécifie les tests de validation, l'organisation et la logistique de validation, ainsi que le plan documentaire. 

\section{Tests unitaires}

Enfin, des tests unitaires permettent de vérifier que chaque fonction est opérationnelle. Nous ne les détaillerons pas tous. Ainsi, nous nous concentrerons sur le critère 1.5.3 "Permettre le clic gauche / la sélection". Nous décrirons la procédure permettant de vérifier le bon fonctionnement d’une partie précise du logiciel, le click gauche. Nous testerons ce module, indépendamment du reste du programme, afin de nous assurer qu’il répond aux spécifications fonctionnelles et qu’il fonctionne correctement en toutes circonstances.
Notre projet comporte un fichier click.cpp où est implémentée la fonction
\begin{lstlisting}
bool closedEyesAndClick(cv::Mat faceROI, CascadeClassifier eyes_cascade)
\end{lstlisting}

Cette fonction permet à la fois de détecter si les yeux sont fermés ou ouverts et de cliquer. Elle fonctionne de la façon suivante (figure \ref{fig:CEAC}) :
\begin{itemize}[font=\tiny, label=\ding{108}]
\item Entrée :
\begin{itemize}[font=\tiny, label=\ding{109}]
\item Le premier argument est \lstinline=faceROI=, de type \lstinline=cv::Mat=. Cet argument est la matrice contenant l’image de l’œil.
\item Le second argument est \lstinline=eyes_cascade=, de type \lstinline=CascadeClassifier=. Cet argument est l’haar\_cascade permettant de détecter un œil.
\end{itemize}
\item  Sortie 1 : La sortie est \lstinline=yeuxFermes=, de type boolean. \lstinline=YeuxFermes= est à \lstinline=true= lorsque l’œil est fermé (pas de détection possible de l’œil) et à \lstinline=false= lorsque l’œil est ouvert. Cette sortie est utilisée par les autres modules de notre partie logicielle.
\item Sortie 2 : Une seconde sortie peut être considérée dans cette fonction, le clic qui agit sur l’OS.
\end{itemize}

\begin{figure}[H]
  \centering
  \includegraphics[scale=0.5]{fonctionClick}
  \caption{Fonction closedEyesAndClick}
  \label{fig:CEAC}
\end{figure}

Nous présenterons donc ici deux tests unitaires, le premier permettant de tester la détection de l’ouverture des yeux et le second permettant de vérifier l’action clic.

\subsection{Test Unitaire 1 : détection de l’ouverture de l’œil}

\paragraph{Les buts/attendus du test}

Le but du test est de vérifier que notre fonction renvoie bien le boolean \lstinline=true= lorsque l’œil est fermé et \lstinline=false= lorsque l’œil est ouvert. 

\paragraph{La préparation du test (mise en place, outils et instrumentation nécessaires)}

Nous avons codé un petit programme (voir annexe \ref{A1}) permettant d’acquérir un flux vidéo et d’appliquer notre fonction à ce flux. Une boucle \lstinline=while= permet, à chaque itération, de transformer l’image du flux en matrice \lstinline=cv::Mat=, de récupérer la sortie de la fonction et de l'afficher. De plus, nous avons fait en sorte qu’un rectangle soit tracé autour de l’œil lorsque celui est détecté, afin de mieux visualiser la détection ou non de l’œil.

\paragraph{Le déroulement du test}

\subparagraph{Conditions initiales (environnement) - Lancement du test}
L’utilisateur est assis à 50 cm de l’écran et de la webcam, positionnés face à lui. Le programme permettant d’effectuer le test unitaire est en cours de fonctionnement.

\subparagraph{Déroulement du test - Fin du test}

\begin{enumerate}
\item L’utilisateur garde les yeux ouverts. Nous vérifions que \lstinline=false= s’affiche dans le terminal et qu’un rectangle apparait bien sur l’image acquise par la webcam.
\item L’utilisateur ferme les yeux. Nous vérifions que \lstinline=true= s’affiche dans le terminal et que le rectangle ne s’affiche plus sur l’image acquise par la webcam.
\end{enumerate}

Ensuite, nous reprenons cette démarche avec l’utilisateur à 1 m de l’écran et de la webcam.

\paragraph{Analyse et consignation des résultats du test}

Après réalisation des tests, nous pouvons conclure que la méthode \lstinline=closedEyesAndClick= fonctionne correctement de 50 cm (comme nous pouvons le voir figure \ref{fig:OeilOuvert} et figure \ref{fig:OeilFerme}) à 1 m. En effet, de 50 cm à 1m, avec un pas de 10 cm, la fonction nous renvoie bien \lstinline=true= lorsque l’œil est fermé et renvoie \lstinline=false= lorsque l’œil est ouvert, tout en dessinant un rectangle autour de celui-ci.

\begin{figure}[H]
  \centering
  \includegraphics[scale=1]{OeilOuvert}
  \caption{Résultat du test unitaire pour la détection de l’ouverture de l’œil – Œil ouvert}
  \label{fig:OeilOuvert}
\end{figure}

\begin{figure}[H]
  \centering
  \includegraphics[scale=1]{OeilFerme}
  \caption{Résultat du test unitaire pour la détection de l’ouverture de l’œil – Œil fermé}
  \label{fig:OeilFerme}
\end{figure}

\subsection{Test Unitaire 2 : clic}

\paragraph{Les buts/attendus du test}

Le but du test est de vérifier que notre fonction permet le clic lorsque l’on ferme l’œil.

\paragraph{La préparation du test (mise en place, outils et instrumentation nécessaires)}

Pour préparer ce test unitaire, nous avons repris le programme du test unitaire précédant. Nous avons néanmoins ajouté la ligne \lstinline=printf("\a");= à la fonction \lstinline=closedEyesAndClick= afin de faire en sorte que l’ordinateur émette un bip lorsque le clic est déclenché.

\paragraph{Le déroulement du test}

\subparagraph{Conditions initiales (environnement) - Lancement du test}

L’utilisateur est assis à 50 cm de l’écran et de la webcam, positionnés face à lui. Le programme permettant d’effectuer le test unitaire est en cours de fonctionnement.

\subparagraph{Déroulement du test - Fin du test}

\begin{enumerate}
\item L’utilisateur garde les yeux ouverts. Nous vérifions que \lstinline=false= s’affiche dans le terminal et qu’un rectangle apparait bien sur l’image acquise pas la webcam.
\item Nous plaçons le curseur sur une fenêtre de l’OS se trouvant en arrière plan.
\item L’utilisateur ferme l’œil concerné par le test.
\item Nous vérifions que le rectangle disparait et que \lstinline=true= s’affiche dans le terminal.
\item Au bout de 1 seconde (minimum), l’utilisateur rouvre l’œil et nous vérifions que le bip sonore est bien émis et que la fenêtre sur laquelle se trouve le curseur passe au premier plan de l’OS. Nous constatons également que \lstinline=false= s’affiche dans le terminal et qu’un rectangle réapparait bien autour de l’œil.
\end{enumerate}

Ensuite, nous reprenons cette démarche avec l’utilisateur à 1 m de l’écran et de la webcam.

\paragraph{Analyse et consignation des résultats du test}

Après réalisation des tests, nous pouvons conclure que la méthode \lstinline=closedEyesAndClick= fonctionne correctement de 50 cm (comme nous pouvons le voir figure \ref{fig:OeilOuvert} et figure \ref{fig:OeilFerme}) à 1 m. En effet, de 50 cm à 1 m, avec un pas de 10 cm, le programme réagit comme prévu (passage de la fenêtre au premier plan).

%----------------------------------------------------------------------------------------
%	PART IV 
%----------------------------------------------------------------------------------------
%\part{Intégration et validation}
%\input{Integration}
%\input{Validation}

%----------------------------------------------------------------------------------------
%	PART V 
%----------------------------------------------------------------------------------------

\chapter{Méthodes de travail}

Dans le cadre de notre projet SPID de l’UV 3.4, nous avons mis en place des méthodes de travail pour ne pas être dépassés par la charge des actions à mener et palier ainsi aux éventuels obstacles.  
L’organisation au sein d’une équipe, pour la réalisation d’un projet en commun, est primordiale pour éviter les échecs.

Nous avons utilisé les premières séances pour apprendre à connaitre chaque membre de l’équipe : connaitre ses méthodes de travail, son rythme, sa tendance à prendre des initiatives ou à suivre des idées, son expérience de leader dans son cursus scolaire, ses disponibilités dans la semaine pour d’éventuelles séances supplémentaires de projet. 
Etant une équipe constituée de 6 membres, nous avons constitué deux sous-équipes. Nous avons tous participé à la mise en place des besoins, fonctions et exigences pendant les premières semaines du projet. 

Le premier sous-groupe, constitué de 3 personnes, s’est penché sur l’état de l’art, afin de trouver différentes pistes pour réaliser le projet. Au fur et à mesure des recherches, nous avons écarté des solutions irréalisables et pris des décisions. Un membre de ce sous-groupe a continué les recherches en parallèle sur une autre façon de procéder. Cela pourrait nous être utile sur le long terme. 
Cette équipe s’est également occupée de la commande du matériel après avoir trouvé la solution la plus adaptée. Ils commencent désormais à faire quelques tests de simulation et de programmation.

La deuxième équipe s'est chargée de la recherche, de l'actualisation et du raffinement des exigences en étroite collaboration avec le premier sous-groupe, car on ne peut pas avancer l’IS sans les informations de choix des solutions. 

Les tâches sont réparties au sein de chaque groupe après un point de début de séance fixant les objectifs journaliers.

\chapter{Outils pour les échanges}

Nous travaillons en groupe, mais chacun fait ses propres recherches avec à la clef un compte rendu de son activité. Pour échanger nos avancées et travaux sur le projet, nous utilisons Git Hub. Depuis les séances "ateliers" de formation à Git Hub, le groupe utilise au mieux ce moyen d'échange pour suivre le projet ainsi que les avancées de chaque membre grâce aux graphes de réseaux pour les dépôts ou modifications. 

En dehors du projet, pour les questions de logistique (horaires de travail, lieux) nous restons en contact par mail. La veille d'un séance, nous fixons la salle et l'horaire, le travail à réaliser ayant été fixé à la fin de la séance précédente, et redéfini en milieu par une réunion d'avancement.  

\chapter{Répartition des tâches dans le temps}

Nous avons mis en place un tableur Excel (voir figure \ref{fig:CarnetBord}). Il nous permet de savoir ce qui était à réaliser la séance précédente et ce que nous devrons faire la séance suivante ; ainsi que le travail réalisé pendant la séance en vue des objectifs fixés. 

\begin{figure}[h]
  \centering
  \includegraphics[scale=0.7]{CarnetBord}
  \caption{Carnet de bord}
  \label{fig:CarnetBord}
\end{figure}

Ainsi chacun peut gérer au mieux l'ampleur de la tache à réaliser en fonction du temps qui lui est imparti.
 
Avant chaque séance, nous faisons un point pour fixer les objectifs du jour, de même en fin de séance pour fixer les objectifs de la prochaine séance. 
En milieu de chaque séance, nous nous réunissons pour voir l'avancement de chacun et redéfinir les tâches si nécessaire. 

\begin{figure}[h]
  \centering
  \includegraphics[scale=0.75]{Gantt}
  \caption{Diagramme de Gantt}
  \label{fig:Gantt}
\end{figure}


%--------
%	PARTIE TECHNIQUE
%--------
\chapter{Partie technique}

Après avoir établi l'IS de notre projet, nous nous sommes lancés dans la réalisation technique au deuxième semestre.

La partie programme de notre système se divise en deux parties. La première, le "détecteur", permet de détecter les pupilles, de déplacer le curseur et de cliquer. La seconde partie, l’interface graphique, propose à l’utilisateur une IHM simplifiée, composée des fonctionnalités qui lui seront les plus utiles. Ce sera l’OS, ici Windows ou MAC, qui permet le lien entre ces deux parties du système. En effet, le détecteur permet de déplacer le curseur et de cliquer sur l’interface graphique.

\section{Le détecteur}

La partie "détecteur" de notre système permet de déplacer le curseur et de cliquer grâce au mouvement oculaire. Comme nous pouvons le voir figure \ref{fig:ESDetecteur}, le détecteur prend en entrée un flux vidéo acquis par la webcam et permet d’agir, en déplaçant le curseur et cliquant, sur l’interface graphique. 

\begin{figure}[H]
  \centering
  \includegraphics[scale=1]{ESDetecteur}
  \caption{Entrée/Sortie du détecteur}
  \label{fig:ESDetecteur}
\end{figure}

\subsection{Principe de fonctionnement}

Le détecteur est composé de différentes parties que nous pouvons retrouver sous forme de méthodes (voir figure \ref{fig:OrganisationDetecteur}). Ces différentes parties permettent de :

\begin{enumerate}
\item Acquérir le flux vidéo et le transformer en images matricielles
\item Détecter le visage
\item Détecter les yeux
\item Déterminer le centre des pupilles
\item Déplacer le curseur en fonction de la position des centres
\item Cliquer
\end{enumerate}

Afin d’expliquer comment sont réalisées ces fonctionnalités, nous allons présenter rapidement différentes méthodes qui composent le détecteur en nous attardant sur les points essentiels vus ci-dessus.

D’autre part, il est important de signaler que nous avons essayé de convertir la position du centre de la pupille en coordonnée de curseur de différentes façons. Ainsi, il existe plusieurs versions informatiques du détecteur. Dans la suite de cette partie du rapport, nous présenterons les méthodes les plus claires afin de faciliter la compréhension du lecteur. Cependant, suivant la version du détecteur, ces méthodes pourront présenter quelques différences.

\begin{figure}[H]
  \centering
  \includegraphics[scale=0.8]{OrganisationDetecteur}
  \caption{Organisation du détecteur}
  \label{fig:OrganisationDetecteur}
\end{figure}

\subsection{Acquisition du flux vidéo et transformation en images matricielles}

La méthode main permet de récupérer le flux vidéo et de le convertir en images matricielles (voir annexe \ref{A2}). Ces images matricielles seront nécessaires aux méthodes utilisées par le détecteur. La méthode \lstinline=cvCaptureFromCam(0)= permet de récupérer le flux vidéo de la webcam par défaut. Ensuite, dans une boucle \lstinline=while=, l’image matricielle, récupérée grâce à la méthode \lstinline=cvQueryFrame=, est inversée selon la verticale grâce à la méthode \lstinline=cv::flip= afin de faciliter l’utilisation des méthodes du détecteur. En effet, sans cette inversion, un mouvement vers la droite de l’utilisateur correspondrait à un mouvement vers la gauche sur l’écran de l’ordinateur. Ainsi, elle facilite par exemple le placement de l’utilisateur.

\subsection{Détection du visage}

Tout d’abord, il faut savoir que nous effectuons la détection du visage et des yeux grâce aux haar\_cascades  \cite{lienhart2002extended}\cite{viola2001rapid}.

\paragraph{Haar\_cascades / Caractéristiques pseudo-Haar}

Les caractéristiques pseudo-Haar (Haar-like features en anglais) sont utilisées en vision par ordinateur pour détecter des objets dans des images numériques. Le principal avantage de celles-ci est la rapidité de calcul. Cependant, cela impose une détection d’objets simples uniquement. Les Haar-like features sont des fenêtres de détection (ou masques) qui délimitent des zones rectangulaires adjacentes. Une caractéristique rectangulaire simple peut être définie comme la différence des sommes de pixels de deux ou plusieurs zones rectangulaires adjacentes. De plus, ce rectangle prend toutes les tailles possibles et se déplace à toutes les positions de l’image à laquelle est appliquée cette caractéristique. Les valeurs indiquent certaines propriétés d’une zone particulière de l’image. \textbf{Une caractéristique est donc un nombre réel qui code les variations du contenu pixellique à une position et taille donnée dans la fenêtre de détection.} Chaque type de caractéristique peut indiquer l’existence (ou l’absence) de certaines particularités dans l’image étudiée telles que des contours ou des changements de texture. Par exemple, une caractéristique à 2 rectangles permet d’indiquer où se situe une frontière entre une zone sombre et une zone claire.

\begin{figure}[H]
  \centering
  \includegraphics[scale=0.8]{pseudoHaar}
  \caption{Un exemple des premières caractéristiques pseudo-Haar utilisées par Viola et Jones en 2001 \\Source : Indif - Wikimedia Commons}
  \label{fig:pseudoHaar}
\end{figure}

Ensuite, un classifieur est créé à partir de quelques centaines d’images de référence de l’objet ciblé (dans notre cas, le visage et les yeux). Ces images sont dimensionnées à la même taille et sont en négatif. Une fois le classifieur créé à partir de cette banque d'images, il peut être appliqué à une région d’une image selon la méthode vue précédemment. Ainsi, il sera retourné un nombre réel proche de 1 lorsque la région ciblée de l’image sera proche du classifieur et 0 lorsqu’elle s’en éloignera. Pour rechercher l’objet dans toute l’image, le classifieur sera déplacé à travers cette dernière. De plus, il sera redimensionné afin de trouver l’objet recherché à différentes tailles (ce qui est plus simple que de redimensionner l’image elle-même). 

Le mot "cascade" dans le nom du classifieur signifie que le classifieur résultant se compose de plusieurs classifieurs simples (étapes) qui sont appliqués à la suite à une région d’intérêt jusqu’à ce que, à un moment donné, la région candidate soit rejetée ou ait passé toutes les étapes. Le mot "boosté" signifie que les classifieurs, à chaque étape de la cascade, sont eux-mêmes complexes et qu’ils sont construits sur des classifieurs de base en utilisant l'une des quatre techniques différentes suivantes : Discrete Adaboost, Real Adaboost, Gentle Adaboost and Logitboost. Les caractéristiques pseudo-haar sont l’entrée du classifieur basique.

\paragraph{Utilisation de ces haar\_cascades}

La méthode \lstinline=detectFace= permet de réaliser la détection du visage grâce à un haar\_cascade. Elle prend en entrée l’image acquise par la webcam. Les haar\_cascades permettant de détecter le visage et les yeux sont d’abord chargés dans le main (voir annexe \ref{A3}) puis, dans la méthode \lstinline=detectFace=, un vecteur de rectangles est créé et rempli lors de l’utilisation de l’haar\_cascade. Dans ce vecteur se trouvent donc les rectangles encadrant le visage (voir annexe \ref{A4}). La méthode trace ensuite un rectangle (voir figure \ref{fig:DetectionVisage}) autour du visage en prenant le premier du vecteur. La méthode \lstinline=detectFace= retourne ce rectangle.

\begin{figure}[H]
  \centering
  \includegraphics[scale=0.8]{DetectionVisage}
  \caption{Détection du visage}
  \label{fig:DetectionVisage}
\end{figure}

\subsection{Détection des yeux}

La méthode \lstinline=closedEyesAndClick= qui prend en argument l’image du visage ainsi que l’haar\_cascade permettant de détecter un œil, et qui retourne un booléen en fonction de l’ouverture ou non de l’œil, permet de détecter l’œil. En effet, de la même manière que pour la détection du visage, un vecteur "eyes" contenant les rectangles encadrant les yeux est créé. Ensuite, la méthode \lstinline=detectMultiScale= permet de remplir ce vecteur grâce à l’image du visage et l’haar\_cascade (voir annexe \ref{A5}).

\subsection{Détermination du centre des pupilles}

Concernant la détection du centre des pupilles, la méthode \lstinline=findEyeCenter= prend en argument une matrice contenant l’image du visage, un rectangle encadrant l’œil et un nom de fenêtre dans laquelle sera affichée l’image de l’œil traitée par la méthode. Tout d’abord, le rectangle encadrant l’œil appliqué à l’image du visage permet de créer une matrice contenant l’image de l’œil. Ce sera grâce à cette matrice que sera détecté le centre des pupilles. De plus, la méthode retourne les coordonnées, dans l’image de l’œil, du centre des pupilles. Il sera ensuite effectué un changement de repère dans la méthode \lstinline=findEyes= afin de d’obtenir les coordonnées des pupilles dans l’image du visage (voir figure \ref{fig:centrePupilles} et annexe \ref{A6}).

\begin{figure}[H]
  \centering
  \includegraphics[scale=0.8]{centrePupilles}
  \caption{Détection du centre des pupilles}
  \label{fig:centrePupilles}
\end{figure}

\paragraph{Localisation du centre des pupilles}

Afin de localiser le centre des pupilles, nous avons utilisé un code de Tristan Hume \cite{eyelike}. Nous pouvons d'ailleurs noter que ce code sera la seule partie de notre projet que nous avons repris d'un système déjà existant. Il s’appuie sur un article de Fabian Timm \cite{timm2011accurate}. Cet article décrit une méthode s’appuyant sur les gradients. La direction de ces gradients est utilisée afin de déterminer le centre des pupilles. Cependant, dans l’image de l’œil, plusieurs centres des pupilles sont possibles si l’on considère seulement les gradients. En effet, certains facteurs tels que les sourcils, les paupières, des lunettes, ou tout simplement des reflets peuvent modifier la valeur théorique des gradients.

Considérons un centre possible $\mathbf{c}$ et un vecteur gradient $\mathbf{g}_i$ à la position $\mathbf{x}_i$. Le vecteur $\mathbf{d}_i$ de la distance normalisée (entre $\mathbf{c}$ et $\mathbf{x}_i$) devrait avoir la même orientation que $\mathbf{g}_i$. Ainsi, le produit scalaire entre ces deux vecteurs devrait être égale à un. Pour trouver le centre optimal $\mathbf{c}*$ des pupilles, il faudrait donc que la somme sur i des produits scalaires entre $\mathbf{d}_i$ et $\mathbf{g}_i$ soit maximale. Ainsi, $c*$ sera égale à l'argument du maximum sur $\mathbf{c}$, noté $arg \max$, (l'ensemble des points en lesquels une expression atteint sa valeur maximale) de cette somme.

\begin{equation}
\mathbf{c}* = arg\max_{\mathbf{c}}\left (\frac{1}{N}\sum_{i=1}^N(\mathbf{d}_i^T\mathbf{g}_i)^2\right )
\label{eq:1}
\end{equation}

\begin{equation}
\mathbf{d}_i = \frac{\mathbf{x}_i - \mathbf{c}}{\left \| \mathbf{x}_i - \mathbf{c} \right \|_2} , \forall i : \|\mathbf{g}_i\|_2 = 1
\label{eq:2}
\end{equation}

Nous pouvons voir une illustration de cette méthode figure \ref{fig:schemaAlgoCentrePupilles}.

\begin{figure}[H]
  \centering
  \includegraphics[scale=1]{schemaAlgoCentrePupilles}
  \caption{Exemple de localisation du centre d'un rond noir sur fond blanc \cite{timm2011accurate}}
  \label{fig:schemaAlgoCentrePupilles}
\end{figure}

\paragraph{Algorithme du gradient}

Afin de calculer les gradients dans l’image de l’œil, l’algorithme utilisé par la méthode \lstinline=findEyeCenter= est celui utilisé par Matlab transformé en C++. En Matlab, cela donne \lstinline=[x(2)-x(1) (x(3:end)-x(1:end-2))/2 x(end)-x(end-1)=], avec x l’entrée (voir annexe \ref{A7} pour la version en C++ utilisée par \lstinline=findEyeCenter=). Le gradient X est ainsi obtenu comme suit \lstinline-cv::Mat gradientX = computeMatXGradient(eyeROI)- et le gradient Y est obtenu en tranposant le résultat de la méthode appliquée à la transposée de X : \lstinline-cv::Mat gradientY = computeMatXGradient(eyeROI.t()).t()-
Ensuite, un seuil est appliqué grâce à la méthode \lstinline=computeDynamicThreshold= afin de filtrer les gradients obtenus \lstinline-double gradientThresh = computeDynamicThreshold(mags, kGradientThreshold)- où "mags" est la matrice des magnitudes de tous les gradients calculés.

\subsection{Déplacement du curseur en fonction de la position des centres}

Cette fonctionnalité est assurée par la méthode \lstinline=calibrateAndMoove=  qui prend en argument la position du centre d’une pupille dans l’image de l’œil. Cette méthode ne retourne rien, mais permet de déplacer le curseur.

\lstinline=CalibrateAndMoove= est composée de cinq parties différentes, quatre consacrées à la calibration et une dernière permettant le déplacement du curseur une fois la calibration effectuée. Les trois premières étapes consistent à enregistrer la position du centre de la pupille lorsque l’utilisateur regarde en haut à gauche de l’écran, puis en haut à droite et enfin en bas à gauche. Ces positions sont stockées dans la matrice \lstinline=pointsMesuresReference[3]=. Ensuite, la quatrième étape permet d’effectuer un changement de repère, la position du centre de la pupille est transformée en position du curseur sur l’écran. Pour cela, nous avons choisi de considérer qu’il s’agit d’un changement de repère affine. Nous pouvons voir figure \ref{fig:calib} un schéma de l’écran avec les coordonnés de trois coins de l’écran dans le repère de l’écran lui-même et dans le repère de l’image de la pupille.

\begin{figure}[H]
  \centering
  \includegraphics[scale=0.6]{Calibration}
  \caption{Schéma calibration}
  \label{fig:calib}
\end{figure}

Avec, entre autre :
\begin{itemize}[font=\tiny, label=\ding{108}]
\item \lstinline-PMR[0] = pointMesureReference[0]- qui correspond aux coordonnées de la pupille lorsque l’utilisateur regarde le coin en haut à gauche.
\item \lstinline-HGE = hautGaucheEcran- qui correspond au coordonnées du point en haut à gauche de l’écran (0,0).
\end{itemize}

Ainsi, la longueur l1 est égale à \lstinline=HGD.x= et correspond à \lstinline=(PMR[1]-PMR[0]).x=

Deux équations sont donc nécessaires à ce changement de repère (une suivant X et une suivant Y) et les coordonnées du point visé sont calculées de la manière suivante :

\begin{lstlisting}
cv::Point pointVise;

double aH = -(double)hautDroitEcran.x / (double)(pointsMesuresReference[0].x - pointsMesuresReference[1].x);
double bH = (double)hautGaucheEcran.x - aH*pointsMesuresReference[0].x;

double aV = -(double)basGaucheEcran.y / (double)(pointsMesuresReference[0].y - pointsMesuresReference[2].y);
double bV = (double)hautGaucheEcran.y - aV*pointsMesuresReference[0].y;

pointVise.x = aH*leftPupil.x + bH;
pointVise.y = aV*leftPupil.y + bV;
\end{lstlisting}

Enfin, la cinquième partie permet de bouger le curseur grâce à \lstinline=SetCursorPos(pointVise.x, pointVise.y)=. De plus, la méthode  sera utilisée pour notre démonstrateur afin de découper l’écran en quatre zones (puis neuf pour les utilisateurs les plus à l'aise avec le système). Ainsi, si le point visé se trouve dans la partie supérieure gauche de l’écran, le curseur sera placé au centre de cette zone (de même que pour les trois autres zones). 

\subsection{Clic}

Le clic est assuré par la méthode \lstinline=closedEyeAndClick= que nous avons déjà présenté précédemment. Lorsque l’œil n’est plus détecté (œil fermé) (détection de l’œil grâce à l’haar\_cascade), un compte à rebours est déclenché. Une fois l’œil rouvert, il est de nouveau détecté et le compte à rebours est stoppé. Si le délai est supérieur à 1 seconde et inférieur à 3 secondes, le clic est déclenché grâce à la fonction \lstinline=mouse_event=.

\begin{lstlisting}
mouse_event(MOUSEEVENTF_LEFTDOWN, 0, 0, 0, 0);
mouse_event(MOUSEEVENTF_LEFTUP, 0, 0, 0, 0);
\end{lstlisting}

\section{Interface graphique}

\subsection{But et enjeux de l’interface}

Dès le début du projet, l’idée d’une interface graphique permettant de contrôler le PC a été envisagée. En effet, la finalité est d’aider les personnes tétraplégiques, or, l’utilisation de l’interface standard d’un ordinateur reste compliquée. Nous avons donc pensé à créer notre propre outil de communication avec l'utilisateur, qui permettrait de lancer les programmes favoris de ce dernier.

\subsection{Critères d’exigence}

Afin de rendre l’IHM la plus ergonomique possible pour l’utilisateur, nous avons décidé de valider deux critères principaux. En premier lieu, l’interface devra être adaptée au contrôle via notre logiciel de gaze-tracking. Celui-ci étant parfois imprécis et pouvant être fatiguant à utiliser sur le long terme, nous avons fait en sorte que les boutons de l’interface occupent le maximum d’espace sur l’écran (voir figure \ref{fig:IG9B}).

\begin{figure}[H]
  \centering
  \includegraphics[scale=0.3]{IG9B}
  \caption{Interface graphique 9 boutons}
  \label{fig:IG9B}
\end{figure}

Deux modes de fonctionnement existent pour notre logiciel de gaze-tracking : un mode de fonctionnement libre, où l’utilisateur peut déplacer le curseur où il veut sur l’écran, et un mode assisté où le mouvement est restreint à certaines zones de l’écran. De plus, deux interfaces distinctes ont été développées. La première à neuf boutons (voir figure \ref{fig:IG9B}), réservée aux utilisateurs les plus à l’aise avec le logiciel de gaze-tracking et la seconde à quatre boutons (voir figure \ref{fig:IG4B}) pour les utilisateurs débutants. Ces deux interfaces autorisent l’utilisation du mode assisté et du mode libre.

\begin{figure}[H]
  \centering
  \includegraphics[scale=0.3]{IG4B}
  \caption{Interface graphique 4 boutons}
  \label{fig:IG4B}
\end{figure}

Dans un deuxième temps, l’interface devait être paramétrable afin de pouvoir lancer les logiciels favoris de l’utilisateur, ou encore un film ou un document PDF. Ainsi, le clic sur l’un des boutons lance un logiciel, soit entré par défaut, soit personnalisable par l’utilisateur. La page permettant de paramétrer les boutons est présentée figure \ref{fig:IGconf9}.

\begin{figure}[H]
  \centering
  \includegraphics[scale=0.3]{IGconf9}
  \caption{Interface de configuration des 9 boutons (existe aussi pour les 4 boutons)}
  \label{fig:IGconf9}
\end{figure}

Chaque chemin peut être modifié à l’aide du bouton "Browse", qui permet de changer le logiciel exécuté par le bouton que l’on a choisi. Ces adresses peuvent ensuite être stockées dans un fichier texte à l’aide du bouton "Save" (le fichier texte se situe dans le dossier contenant l’exécutable). En cas d’erreur ou d’une manipulation non souhaitée, le bouton "Default" permet de lier de nouveau le bouton erroné à un chemin valide. Ces adresses par défaut permettent de lancer les logiciels d’ergonomie de Windows, ainsi que les utilitaires classiques tels que la calculatrice ou Internet Explorer.

\subsection{Outils de réalisation}

Comme langage de programmation, nous avons choisi le C++ couplé à Qt Creator, pour réaliser cette interface. En effet, notre programme de gaze-tracking étant codé en C++, il sera plus facile dans le futur de réunir les deux logiciels en un seul (chose qui n’est pas encore faite à ce jour). D’autre part la librairie Qt permet de faire rapidement et simplement des interfaces graphiques, et c’est donc présentée comme une évidence.

\section{Limites du système et problèmes rencontrés}

\subsection{Problèmes matériels}
L'installation d'OpenCV sur Windows n'est pas aisée et a fait perdre beaucoup de temps à deux membres de l'équipe, déjà fortement réduite.
De plus, ayant besoin de fixer la tête pour que notre calibrage soit efficace, nous avons du trouver un moyen de la maintenir. Nous avons finalement opté pour une mentonnière, qui nous a gentiment été prêtée par un ophtalmologue.

\subsection{Limites du détecteur}

Concernant les limites de la partie détecteur de notre système, nous pouvons distinguer trois grandes catégories :
\begin{itemize}[font=\tiny, label=\ding{108}]
\item Limites liées au mouvement de la tête
\item Limites liées à la caméra
\item Limites liées à la luminosité
\end{itemize}

\subsubsection{Limites liées au mouvement de la tête}

La principale limite de notre démonstrateur est l’absence de possibilité de mouvement de la tête. En effet, nous devons utiliser une mentonnière afin de maintenir la tête et de réduire les mouvements de celle-ci. Cette restriction de notre démonstrateur peut s’expliquer par trois limites de notre système.

\paragraph{Limites des haar\_cascades}

La détection du centre des pupilles a d’abord été testée en utilisant l’image du visage où étaient récupérées les images des yeux grâce à des constantes liées à la morphologie du visage. Cependant, le rectangle encadrant le visage, renvoyé lors de l’utilisation de l’haar\_cascade permettant la détection du visage, n’est pas suffisamment précis. En effet, nous avons constaté un léger décalage de quelques pixels au cours du temps de ce rectangle (comme une vibration). Or, les coordonnées du centre des pupilles ayant une variation de seulement quelques pixels, ces décalages du rectangle modifiaient trop les coordonnées du centre pour avoir une estimation correcte du point visé sur l’écran. Pour corriger le problème, nous avons tout d'abord fixé le cadre de la tête lorsque sa variation restait faible. Néanmoins, même si c'était mieux, un très petit mouvement de la tête obligeait à recalibrer le système. L'idée de mettre une gommette de couleur vive sur le visage de l'utilisateur a alors été émise. En effet, il est facile de filtrer une couleur avec OpenCV, ce qui nous donnait un point fixe qui servait de repère. Cette solution n'a pas non plus abouti. 
Nous avons finalement essayé de récupérer l’image des yeux non plus grâce à des constantes, mais en utilisant un haar\_cascade permettant de récupérer le rectangle encadrant les yeux. Cependant, encore une fois, la position de ce rectangle variait trop au cours du temps. C’est pourquoi notre démonstrateur n’effectue pas cette détection du visage mais seulement celle de l'oeil et filme directement celui-ci avec une tête immobile.

\paragraph{Orientation de la tête}

Ensuite, une deuxième limite à notre système est qu’il ne prend pas en compte l’orientation de la tête. Si le déplacement du visage dans le plan était pris en compte, un mouvement circulaire de la tête vers la gauche ou la droite provoquerait un dérèglement du calibrage. En effet, lorsqu’un utilisateur fixe un point de l’écran en tournant la tête ou non, cela change les coordonnées du centre des pupilles. Cependant, nous pouvons penser qu’un utilisateur tétraplégique ne bougera pas sa tête, mais il ne faudra pas non plus que sa tête glisse.

\paragraph{Tête verticale seulement}
La dernière limite quant  au mouvement de la tête est que l’haar\_cascade ne permet de détecter qu’un visage plus ou moins vertical. Empiriquement, nous avons vu qu’à partir d’un angle d’environ 10 degrés, l’haar\_cascade ne permettait plus la détection du visage. 

\subsubsection{Limites liées à la caméra}

Pour notre démonstrateur, nous utilisons une mentonnière afin de limiter les mouvements du visage qui induiraient de mauvaises valeurs pour les points visés. En effet, un décalage du visage par rapport à la caméra induirait un étalonnage obsolète. Ainsi, le principe du casque / lunettes étudié lors de la phase bibliographique aurait peut-être permis de pallier ce problème.

\subsubsection{Limites liées à la luminosité}

Une des principales limites de l’utilisation de la méthode permettant la détection du centre des pupilles est qu’elle s’appuie sur les gradients. Or, ces gradients peuvent être plus ou moins en accord avec leur valeur théorique selon la luminosité. En effet, un simple reflet dans la partie de l’image de la pupille peut avoir une influence assez conséquente sur la détermination de son centre. Si la méthode utilisée permet de réduire les défauts liés à ces aléas, les centres calculés restent parfois imprécis (hésitation entre deux positions par exemple). De plus, de part l’utilisation de la méthode des gradients, la méthode de détection du centre des pupilles est assez robuste avec des yeux clairs (yeux bleus), mais reste moins efficace avec les yeux sombres (yeux marrons).

\subsection{Limites d'utilisation}

\subsubsection{Limite d’utilisation pour l’utilisateur}

Les tests de validation fonctionnelle (un exemple de fiche complétée est donné en annexe \ref{AFV}) nous ont permis de mettre au clair plusieurs limitations concernant l’utilisateur de notre système. Tout d’abord, notre système reste encore trop fatigant lorsqu’il est utilisé en mode libre. En effet, les utilisateurs se sont vite fatigués à essayer de naviguer sur l’écran lorsque celui-ci n’est pas prédécoupé en différentes zones. La navigation est de plus encore difficile car trop imprécise.

D’un point de vue du confort, l’immobilité de la tête due à la mentonnière est aussi un point à améliorer. Cette immobilité forcée empêche l’utilisateur de parler ou de bouger la tête sous peine de devoir refaire une calibration. Le fait de fixer la mentonnière et l’ordinateur à la table pourrait aider à améliorer cette situation en offrant des repères parfaitement fixes.

Le champ de vision de l’utilisateur pose aussi un problème. Le fait de se concentrer sur une zone particulière de l’écran afin de cliquer dessus empêche la vision globale de l’écran.

La morphologie de l’utilisateur parait enfin très déterminante sur les résultats obtenus. Les yeux plus clairs semblent permettre un meilleur contrôle de la souris en comparaison aux yeux foncés. Certaines morphologies du visage rendent aussi la détection des yeux difficiles, notamment lorsque l’utilisateur observe les angles de l’écran.

\subsubsection{Limite d’utilisation de l’interface graphique}

Concernant l’interface graphique, la principale limitation vient du fait qu’elle n’est pas encore totalement adaptée au contrôle avec notre logiciel. En effet le découpage de l’écran en différentes zones distinctes oblige à sectionner chaque onglet de façon différente. L’idéal serait donc de pouvoir faire abstraction de ce découpage et laisser un total contrôle à l’utilisateur. Cependant, la précision de notre logiciel ne le permet pas encore.

Une seconde limitation est liée à la personnalisation de notre interface. L’utilisateur ne peut pas configurer lui-même le nombre de bouton ou encore les placer où il veut à l’écran. De plus, dans le cas où l’utilisateur est tétraplégique, la personnalisation du logiciel lancé par un bouton donné devra forcément être faite par une tierce personne.

A l’heure actuel, notre interface ne permet pas de lancer deux logiciels en même temps ce qui pose problème pour le multi-tasking. Cependant ce problème pourra facilement être corrigé dans les versions futures.

La dernière limitation vient du fait que même si notre interface permet facilement de lancer un logiciel, celui-ci doit aussi être utilisable à l’aide du gaze-tracking. Or la majorité des logiciels les plus courants ne possèdent pas de telles options. Il faudrait donc créer un lot de logiciels adaptés à cette utilisation et notamment dans le cas où l’utilisateur est tétraplégique.

%----------------------------------------------------------------------------------------
%	PART VI 
%----------------------------------------------------------------------------------------
%\chapter{Choix et justifications}

\section{Émetteurs infrarouges}

Concernant les émetteurs infrarouges, des LEDs pourront être utilisées. En effet, la distance écran/utilisateur ne sera pas très grande, et de simple LEDs seront suffisantes pour éclairer la pupille de l'utilisateur. Nous pourrons par exemple utiliser l’Émetteur infrarouge (IR) Kingbright L-934SF4BT (880 nm 50 ° 3 mm) ou alors l’Émetteur infrarouge (IR) Harvatek HT-159IRAJ 940 nm 20 ° 1206 CMS coûtant respectivement 0.90€ et 0.35€ pièce. Le choix se fera en fonction des connectiques nécessaires (la première LED offre une sortie radiale avec des connecteurs UY2, alors que la seconde se présente sous forme de puce).

\section{Caméras}

Si de nombreux types de caméras sont possible, nous avons retenu la webcam. En effet, la résolution des webcams HD actuelles est largement suffisante pour une détection de pupille. De plus, elles sont facilement utilisables via des ordinateurs, c'est-à-dire simple d'installation et il est aisé de récupérer les flux vidéos. Leur connexion USB facilite elle aussi leur utilisation. Nous pouvons également noter que les webcams sont relativement peu coûteuses.


\subsection{Caméras «grand-angle»}

Pour le suivi du visage, la Webcam LOGILINK USB avec LED a été retenue. Elle est compatible avec Windows, Mac et Ubuntu et ne posera donc pas de problème de compatibilité. De plus elle possède un système lui permettant de pivoter sur 360° permettant ainsi le meilleur suivi du visage possible. Cependant, d'autres webcams présentent ces avantages et le détail qui nous a décidé dans ce choix est la présence de LEDs (voir figure \ref{fig:LEDCam}). En effet, trois LEDs sont présentes de chaque coté de l'objectif et pourraient éventuellement être remplacées par des émetteurs infrarouges.

\begin{figure}[H]
  \centering
  \includegraphics[scale=0.5]{LEDCam}
  \caption{Présence de LEDs sur la webcam}
  \label{fig:LEDCam}
\end{figure}

Le placement des LEDs sur la webcam pourrait nous éviter de devoir mettre en place un système pour pouvoir brancher ces LEDs.


\subsection{Caméras «petit-angle»}

Ensuite, concernant la caméra petit-angle, la Webcam Trust Widescrenn Full HD 1080p (voir figure \ref{fig:CamPetitAngle}) a été choisie. D'abord parce que le tacking des pupilles demande une grande résolution et que cette webcam est adapté à la résolution Full HD 1080p (1920x1080), mais aussi parce qu'elle intègre une éclairage LED permettant, comme précédemment, de placer nos émetteurs infrarouges.

\begin{figure}[H]
  \centering
  \includegraphics[scale=0.5]{CamPetitAngle}
  \caption{Webcam Trust Widescrenn Full HD 1080p}
  \label{fig:CamPetitAngle}
\end{figure}Émetteur infrarouge




\chapter{Résultats et analyses}

analyse des tests et des performances
analyse des échecs, des décalages et des retards
Que reste-t-il à faire ? Comment ?



% CONCLUSION
\newpage
\addcontentsline{toc}{part}{Conclusion}
\chapter{Conclusion}


Ce projet de l'U.V. 3.4 nous permet de comprendre les étapes qui mènent à l'aboutissement d'un projet. En effet, afin de se faire une idée de ce qui est réalisable, nous avons tout d'abord procédé à un état de l'art. Cette étude préalable nous a permis de comprendre que de nombreuses technologies existent déjà afin d'effectuer de l'eye tracking. En comparant les possibilités offertes avec nos attentes, nous avons décidé que la technologie la plus pertinente serait la suivante :

- Filmer l'utilisateur à l'aide d'une première caméra. Celle-ci repère la position du visage dans l'ensemble de l'image, et transmet la position à une deuxième caméra.

- Zoomer sur le visage à l'aide de la deuxième caméra qui est infra-rouge. Pour cela nous installerons  un filtre sur une caméra normale afin de diminuer les coûts. Celle-ci pourra alors détecter la pupille de l'utilisateur et déterminer de manière précise ses mouvements.

\vspace*{1cm}

A partir de cette décision nous avons pu mettre en place un dossier fonctionnel. Dans un premier temps, l'approche Top-Down nous a permis d'identifier les exigences de notre système. Nous avons défini une fonction  principale, deux fonctions de service et deux fonctions de contraintes. Ces exigences ont ensuite été caractérisées par une approche Bottom-Up. Nous les avons regroupées par fonctions principales, et raffinées en FAST. Nous sommes restés le plus général possible dans les intitulés des fonctions pour qu'elles soient adaptables si l'on était amenés à changer notre système. Enfin nous avons spécifié les données et proposé une architecture fonctionnelle. Cependant, ces définitions pourront être amenées à changer.

\vspace*{1cm}

En effet, concernant la partie Ingénierie des Exigences, nous avons pris conscience que la mise en place de l'architecture fonctionnelle n'est pas un exercice aisé compte tenu de l'objectif de notre projet. Celui-ci offre une grande liberté de réalisation avec un grand choix de méthodes, ce qui ne nous permet pas à l'heure actuelle de connaître précisément l'architecture de notre futur système. Notre projet demande une avancée dans la conception physique (début d'algorithme) afin de pouvoir caractériser correctement l'architecture fonctionnelle. Tant que le choix de notre système n'est pas validé, le dossier fonctionnel pourra évoluer. Nous avons donc décidé de nous consacrer désormais à l'obtention d'un eye tracking performant. En effet, à ce stade du projet, si nous parvenons à suivre les yeux d'un utilisateur, ce flux vidéo reste trop saccadé. L'étape suivante sera de fluidifier ce résultat afin d'effectuer une détection de la pupille et de valider notre système.




\appendix
%\part{Annexes}
\newpage
\addcontentsline{toc}{part}{Annexes}
\chapter{Plan de validation fonctionnelle}
\label{APV}
\includepdf[pages={1-8}]{PlanValidationFonctionnelle.pdf}

\chapter{Programme permettant d’acquérir un flux vidéo}
\label{A1}
Des variables globales telles que \lstinline=eyes_cascade_name= ne sont pas indiquées ici. Elle n’aident pas à la compréhension du code et nécessitent certaines informations propres à l’ordinateur utilisé (chemin de répertoire par exemple).

\begin{lstlisting}
int main(int argc, const char** argv) {

	CvCapture* capture;
	cv::Mat frame;

	// Load the cascade
if (!eyes_cascade.load(eyes_cascade_name)){ printf("--(!)Error loading\n"); return -1; };
	capture = cvCaptureFromCAM(0);
	if (capture) {
		while (true) {
			frame = cvQueryFrame(capture);
			// mirror it
			cv::flip(frame, frame, 1);
			frame.copyTo(debugImage);

			// closedEyesAndClick function
			if (!frame.empty()) {
				bool res = closedEyesAndClick(frame, eyes_cascade);
				printf("%s\n", res ? "true" : "false");
			}
			else {
				printf(" --(!) No captured frame -- Break!");
				break;
			}
			imshow(main_window_name, debugImage);
			int c = cv::waitKey(10);
			if ((char)c == 'c') { break; }
			if ((char)c == 'f') {
				imwrite("frame.png", frame);
			}

		}
	}

	releaseCornerKernels();

	return 0;
}
\end{lstlisting}

\chapter{Code permettant de récupérer le flux vidéo et de le convertir en images matricielles}
\label{A2}
\begin{lstlisting}
CvCapture* capture;
cv::Mat frame;
capture = cvCaptureFromCAM(0);
	if (capture) {
		while (true) {
			frame = cvQueryFrame(capture);
			// mirror it
			cv::flip(frame, frame, 1);
			frame.copyTo(debugImage);
\end{lstlisting}

\chapter{Haar\_cascades permettant de détecter le visage et les yeux}
\label{A3}
\begin{lstlisting}
cv::String eyes_cascade_name = "CheminDossier/haarcascade_eye_tree_eyeglasses.xml";
cv::CascadeClassifier eyes_cascade;
cv::String face_cascade_name = " CheminDossier/haarcascade_frontalface_alt.xml";
cv::CascadeClassifier face_cascade;
// Load the cascades
if (!face_cascade.load(face_cascade_name)){ printf("--(!)Error loading face cascade, please change face_cascade_name in source code.\n"); return -1; };
if (!eyes_cascade.load(eyes_cascade_name)){ printf("--(!)Error loading eyes cascade, please change eyes_cascade_name in source code.\n"); return -1; };
\end{lstlisting}


\chapter{Matrice contenant les rectangles encadrant le visage}
\label{A4}
\begin{lstlisting}
cv::Rect detectFace(cv::Mat frame) {

	std::vector<cv::Rect> faces; //vecteur contenant les rectangles encadrant le visage

	std::vector<cv::Mat> rgbChannels(3);
	cv::split(frame, rgbChannels);
	cv::Mat frame_gray = rgbChannels[2];
	//-- Detect faces
	face_cascade.detectMultiScale(frame_gray, faces, 1.1, 2, 0 | CV_HAAR_SCALE_IMAGE | CV_HAAR_FIND_BIGGEST_OBJECT, cv::Size(150, 150)); //utilisation de l'haar_cascade afin de remplir faces

	if (faces.size() > 0) {
		rectangle(debugImage, faces[0], 1234); // on trace un rectangle autour du visage
	}
	else faces.push_back(cv::Rect(0, 0, 0, 0));

	return faces[0];

}
\end{lstlisting}

\chapter{Méthode detectMultiScale}
\label{A5}
\begin{lstlisting}
void closedEyesAndClick(cv::Mat faceROI, CascadeClassifier eyes_cascade)
{
	std::vector<cv::Rect> eyes;
	eyes_cascade.detectMultiScale(faceROI, eyes, 1.1, 2, 0 | CV_HAAR_SCALE_IMAGE, cv::Size(150, 150)); // detection yeux
\end{lstlisting}


\chapter{Changement de repère des coordonnées du centre des pupilles dans la méthode findEyes}
\label{A6}
\begin{lstlisting}
// change eye centers to face coordinates
	rightPupil.x += rightEyeRegion.x;
	rightPupil.y += rightEyeRegion.y;
	leftPupil.x += leftEyeRegion.x;
	leftPupil.y += leftEyeRegion.y;
\end{lstlisting}

\chapter{Algorithme du gradient}
\label{A7}
\begin{lstlisting}
cv::Mat computeMatXGradient(const cv::Mat &mat) {
	cv::Mat out(mat.rows, mat.cols, CV_64F);

	for (int y = 0; y < mat.rows; ++y) {
		const uchar *Mr = mat.ptr<uchar>(y);
		double *Or = out.ptr<double>(y);

		Or[0] = Mr[1] - Mr[0];
		for (int x = 1; x < mat.cols - 1; ++x) {
			Or[x] = (Mr[x + 1] - Mr[x - 1]) / 2.0;
		}
		Or[mat.cols - 1] = Mr[mat.cols - 1] - Mr[mat.cols - 2];
	}

	return out;
}
\end{lstlisting}

\chapter{Fiche de validation fonctionnelle complétée}
\label{AFV}
\includepdf[pages={1-2}]{FicheValidationFonctionnelle.pdf}

%----------------------------------------------------------------------------------------
%	BIBLIOGRAPHIE
%----------------------------------------------------------------------------------------
\newpage
\addcontentsline{toc}{part}{Bibliographie}
\bibliographystyle{plain-fr}
\bibliography{bibliographie}
\nocite{*}
%----------------------------------------------------------------------------------------
%	INDEX
%----------------------------------------------------------------------------------------
%\cleardoublepage
%\phantomsection
%\setlength{\columnsep}{0.75cm}
%\addcontentsline{toc}{part}{Index}
%\label{sec:index}
%\printindex

%----------------------------------------------------------------------------------------
%	GLOSSAIRE
%----------------------------------------------------------------------------------------
%\cleardoublepage
%\phantomsection
%\setlength{\columnsep}{0.75cm}
%\addcontentsline{toc}{part}{Glossaire}
%\printglossaries

%----------------------------------------------------------------------------------------

\end{document}